\appendix
\newpage

\chapter{Termo de Abertura do Projeto - Modelo}

\begin{description}
	
	\item[Nome do projeto:] Curso de extensão em Gestão de Projetos
	
	\begin{quote}
	\emph{Identificação única do projeto durante todo seu ciclo de vida.}	
	\end{quote}
	
	\item[Propósito ou justificativa do projeto:] Existe uma grande demanda por projetos nas organizações, sejam públicas ou privadas, em todas os setores econômicos. Projetos permitem a implementação do planejamento estratégico, a criação de novos produtos e serviços e o desenvolvimento e execução de ações de melhorias.

	Em contrapartida, há uma carência no mercado de profissionais capacitados para gerenciar estes projetos e garantir seu sucesso mantendo prazos, custos e qualidade.

	Estas razões justificam a criação do curso de extensão em gestão de projetos.
	
	\begin{quote}
	\emph{Razão de negócio para o qual o projeto é a solução. Deve ser claro e fazer com o que a equipe entenda a razão de existir o projeto.}
	\end{quote}

	\item[Objetivos:] capacitar em 10 semanas até 30 participantes, entre alunos e profissionais, com conceitos e ferramentas das melhores práticas de gerenciamento de projetos na visão do \pmi.
	
	\begin{quote}
		\emph{O que se pretende alcançar com o projeto. Não confundir com o que será criado pelo projeto. Deve ser ``SMART'' - específico (evitar definições vagas), mensurável (como conseguirei medir os resultados e verificar se alcancei meu objetivo?), alcançável (objetivos impossíveis ou muito difíceis de se alcançar podem desmotivar a equipe), relevalente (importantes para a organização e para a equipe) e temporal (deve ter uma data ou período para verificação do seu sucesso).}
	\end{quote}

	\item[Requisitos de alto nível] : 
	
		\begin{itemize}
			
			\item O curso deve atender não somente os alunos da instituição mas também profissionais e outras pessoas interessadas em gestão de projetos;
			
			\item O curso deve integrar o conhecimento teórico com a prática através de oficinas realizadas no laboratório de informática.
			
		\end{itemize}
		
	\begin{quote}
		\emph{Lista de elementos que o produto ou serviço criado pelo projeto deve possuir. Podem incluir as necessidades, desejos e expectativas do patrocinador, cliente e outras partes interessadas. Não confundir com pré-requisito! Pré-requisito é o que precisa antes do projeto começar. Requisito é o que precisa acontecer ao final (entrega) do projeto.}
	\end{quote}
	

	\item[Descrição do projeto em alto nível:] curso de gestão de projetos contendo 10 encontros onde serão abordadas as melhores práticas de acordo com o \bok do \pmi.

	\begin{quote}
		\emph{Descrição do que o projeto realizará}
	\end{quote}

	\item[Premissas: ] todos os computadores do laboratório devem possuir o pacote Microsof Office e o Microsoft Project.
	
	\begin{quote}
		\emph{Premissa é qualquer hipótese ou proposição que é considerada verdadeira para se dar continuidade ao projeto. Nem sempre é possível validar uma premissa antes de se iniciar o projeto e, para evitar que o mesmo seja paralisado, considera-se a premissa verdadeira até que ela possa ser verificada ou se mostre falsa. Obrigatoriamente toda premissa é um risco e é uma boa prática o registro de todas elas no Gerenciamento de Riscos.}
	\end{quote}

	\item[Restrições: ] o horário de funcionamento do laboratório é de 08:00 às 17:00.
	
	\begin{quote}
		\emph{São as condições que afetarão o desempenho do projeto ou de um processo. Podem ser internas ou externas. São limitações impostas e que não podem ser negociadas.}
	\end{quote}

	\item[Riscos de alto nível]:
	
	\begin{itemize}
		\item Devido à grande quantidade de participantes, caso haja uma retenção muito pequena do conteúdo pode haver problemas para tratar todas as dúvidas;
		\item Pelo fato dos encontros serem no laboratório de informática, é possível que os participantes dispersem sua atenção ao navegar pela internet.
	\end{itemize}

	\begin{quote}
		\emph{Aqui são documentados os primeiros riscos identificados do projeto. Ainda não é o momento de traçar os planos de ação para lidar com eles mas sim conscientizar a equipe e demais partes interessadas que eles existem e devem ser cuidados durante todo o ciclo de vida do projeto.}
	\end{quote}

	\item[Resumo do cronograma de marcos]:
	
		\begin{itemize}
			\item \textit{data 1:} início do curso;
			\item \textit{data 2:} avaliação dos participantes;
			\item \textit{data 3:} final do curso;
			\item \textit{data 4:} entrega da avaliação de desempenho do curso;
		\end{itemize}

	\begin{quote}
		\emph{Datas em que os principais eventos (marcos) do projeto ocorrerão. Neste momento ainda não há necessidade de detalhamento do cronograma, devendo o gerente de projetos se concentrar nas principais fases ou entregas.}
	\end{quote}

	\item[Resumo do orçamento]:
	
		\begin{table}[h!]
			\centering
			\begin{tabular}
				{
					|l|r|
				}
				
				\hline
				Instrutor&R\$100,00\\
				\hline
				Material de apoio&R\$20,00\\
				\hline
				Total&R\$120,00\\
				\hline
				
			\end{tabular}
		\end{table}
	

		\begin{quote}
			\emph{Estimativas ou valores pré-definidos (contratos, licitações, etc.) devem ser especificados de forma resumida e se concentrando nos principais elementos, fases ou entregas.}
		\end{quote}

	\item[Critérios de aceitação] :
		
		\begin{table}[h!]
			\centering
			\begin{tabular}
				{
					|l|l|
				}
				
				\hline
				Lista de presença&Devem estar assinadas pelos participantes e pelo instrutor\\
				\hline
				Avaliação dos participantes&Deve ser entregue sem rasuras e com a assinatura do instrutor\\
				\hline
				
			\end{tabular}
		\end{table}
		
		
		\begin{quote}
			\emph{Relacione aqui as entregas que deverão ser avaliadas e quais os critérios que serão utilizados para o seu aceite.}
		\end{quote}

	\item[Gerente do projeto]: Gladistone M. Afonso, responsável pelo acompanhamento do projeto, porém não estará autorizado a contratar, realizar modificações no escopo ou ampliar o orçamento já autorizado. Ele deverá se reportar ao Departamento de Extensão para autorizar as solicitações de mudança.
	
	\begin{quote}
		\emph{Um dos principais objetivos do TAP é a designação do gerente de projetos e a determinação de seus poderes dentro do projeto.}
	\end{quote}

\end{description}