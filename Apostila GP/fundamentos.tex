% !TEX root = Apostila GP.tex

\capitulo{Fundamentos}

O Guia do Conhecimento em Gerenciamento de Projetos (\bok) é uma norma reconhecida para a profissão de gerenciamento de projetos. Um padrão é um documento formal que descreve normas, métodos, processos e práticas estabelecidas. Assim como em outras profissões como advocacia, medicina e contabilidade, o conhecimento contido nesse padrão evoluiu a partir das boas práticas reconhecidas de profissionais de gerenciamento de projetos que contribuíram para o seu desenvolvimento.

\secao{O que é um projeto?}

De acordo com o \bok: "Projeto é um esforço temporário empreendido para criar um produto, serviço ou resultado único." \citep[p. 3]{pmbok}

Esforço temporário significa que um projeto sempre tem início e término determinados. Porém, o termo ``temporário'' não significa que o projeto será sempre de curto prazo.

Existem 4 condições para determinar o término de um projeto:

\begin{itemize}
	\item Os objetivos do projeto foram atingidos;
	\item Os objetivos do projeto não serão ou não podem ser atingidos;
	\item A necessidade do projeto deixa de existir;
	\item O cliente, patrocinador ou financiador deseja encerrá-lo.
\end{itemize}

O produto, serviço ou resultado único gerado pelo projeto pode ser tangível ou intangível. O projeto pode gerar componentes ou aprimoramentos de outros itens, serviços, melhorias ou desenvolvimento de novos produtos e serviços, documentos, mudança na estrutura, processos, pessoal ou estilo de uma organização, desenvolvimento ou aquisição de sistema de informações, registro de esforço de pesquisa, construções, entre outros.