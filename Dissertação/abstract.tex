Software development has become one of the most important activities in today's world technology scenario, because applications and information systems permeate all economic activities, reaching, or at least afecting, all social spheres. Small and medium software delevopment enterprises play an important role in this scenario and represent a big slice of the industry. However, their processes are not formalized, causing negative consequences as late deliveries, budget blow-out, and conflicts with customers. The adoption of good practices in project management and software development through introduction of technical standards, such as \iso, created specifically for small and medium software developers, is the ideal path for the profissionalization of the industry. In turn, the process of adopting this or any other standard comes with a high cost, in terms of time and financial and human resources, making its execution very dificult. This work aims to create a process of self diagnosis that allows the assessment of the maturity level of a company and the selection of the first steps in the adoption of \iso, considering individual company values in order to generate results in the short run that could be perceived as aligned with those values, becoming more atractive, stimulating the team and directors, and preventing an early drop out of the implementation project.

Keywords: ISO 29110, technical standards, software development.