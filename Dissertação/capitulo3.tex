% !TEX root = dissertacao.tex
\chapter{Considera��es Iniciais Sobre a Pesquisa}
\label{trabalhosrelac}

%\section{Concluso do Captulo}
%	Este captulo teve como principal valor bibliogrfico a apresentao dos conceitos iniciais e das abordagens que esto sendo desenvolvidas dentro do ambiente de nuvens computacionais, coube principalmente nesse sentido:\\
%	
%	\begin{itemize}
%	\item Definir nuvens computacionais e os tipos de servios prestados por essas infraestruturas, bem como o modelo de prestao de servios computacionais adequados ao usurio final;
%	\item Apresentar os conceitos de segurana na nuvem, escalabilidade e virtualizao;
%%	\item Apresentar os requisitos necessrios para coleta de recursos e como o modelo se encaixa nesse trabalho;
%	\item Mostrar os requisitos e objetivos do teste de desempenho;
%	\item Distinguir aplicaes comerciais de cientficas;
%	\item Apresentar as definies de dependabilidade e afinidade;
%	\item Caracterizar os \textit{Dwarfs};
%	\item Apresentar os trabalhos relacionados  proposta sendo apresentada.
%	\end{itemize}
%	
%	O trabalho proposto aqui precisa ser demonstrado sobre a infraestrutura de nuvens computacionais e esse  o principal motivo da explanao desse conceito, at o nvel de segurana, escalabilidade e virtualizao do ambiente. Como a apresentao do tema  baseado em afinidade entre recursos de hardware e os softwares executados nessa infraestrutura, faz-se necessrio o entendimento desses conceitos e de como eles sero includos no trabalho proposto. Por fim, foi feita uma pesquisa bibliogrfica, buscando principalmente trabalhos similares ao proposto, para justificar sua apresentao na tese de doutorado. Desta pesquisa tomou-se que at o momento, no foram encontrados trabalhos similares ao proposto.\\
%	
%	
