% !TEX root = Monografia Mestrado.tex

\chapter{Introdução}

\section{Introdução}

O autor identificou na empresa DataSerra, uma pequena empresa de desenvolvimento de \sw, várias demandas de melhoria em seus processos. Anteriormente, com o objetivo de alcançar essas melhorais, o autor havia pesquisado e aplicado alguns padrões encontrados na \iso mas vários fatores levaram à paralisação e até mesmo abandono dos poucos processos que viam sendo implementados.

A \iso foi desenvolvida especificamente para pequenas empresas ou pequenas equipes de desenvolvimento de \sw, chamadas em inglês de \textit{Very Small Entities} (VSE). O autor pretende dar continuidade à implementação de seus processos utilizando o trabalho de pesquisa da dissertação como catalisador, assim como a carga de conhecimento adquirida durante o curso de Mestrado.

Como cita \cite{pressman}, a construção de \sw dentro de prazos estabelecidos e com qualidade ainda é um problema que atinge grande parte das empresas desenvolvedoras. Um dos maiores problemas envolvendo prazos e qualidade na empresa foco foi a demora ou até mesmo inexistência de \textit{feedback} aos clientes em relação à solução de problemas ou implementação de solicitações de mudanças. O autor selecionou este problema como foco do trabalho de dissertação 

\section{Objetivos gerais}

O trabalho de dissertação tem como objetivo \textbf{criar processos e ferramentas de \sw que gerenciem a comunicação com o cliente}.

\section{Objetivos específicos}

Para lograr êxito na criação da solução ideal para a demanda identificada, o autor determinou os objetivos abaixo:

\begin{itemize}

\item Realizar um diagnóstico da situação atual dos processos que envolvem comunicação com o cliente;

\item Categorizar a situação atual destes processos através da análise SWOT\footnotemark;

\footnotetext{Avaliação das forças, fraquezas, oportunidades e ameaças, dos termos em inglês \textit{strengths, weaknesses, opportunities e threats} \citep{kotler}}

\item Pesquisar a norma \iso a procura de soluções que possam ser aplicadas às ameaças e fraquezas identificadas na análise SWOT;

\item Desenvolver um projeto de melhoria de processo que contemple a criação de ferramentas de \sw que facilitem a implantação e operacionalização das melhorias;

\item Executar o projeto de melhorias;

\item Coletar os resultados do projeto de melhorias.

\end{itemize}

\section{Metodologia}

O trabalho se dará em tres etapas distintas: análise, projeto e conclusão.

A análise se dará a partir do mapeamento dos processos de produção atuais e olhar crítico em cada uma das etapas identificadas. O autor se utilizará de algumas ferramentas de administração, tais como a análise SWOT, para caracterizar e avaliar cada etapa do processo produtivo.

O projeto será a etapa onde as ações necessárias para a melhora dos processos serão elencadas, priorizadas e devidamente documentadas. Também serão identificados os principais atores (\textit{stakeholders}), o cronograma, os recursos necessários e outros elementos, conforme diretrizes do PMBOK\footnotemark{} criado pelo PMI\footnotemark. Nesta etapa serão desenvolvidas as ferramentas de \sw que buscarão facilitar as melhorias através da integração de informações vitais para os processos identificados.

\footnotetext{\textit{Project Management Book Of Knowledge}, guia de boas práticas de gerência de projetos mundialmente reconhecido}
\footnotetext{\textit{Project Management Institute}, instituição internacional referência em gerência de projetos.}

A conclusão consistirá na coleta dos resultados obtidos após a implantação do projeto de melhoria dos processos e análise da qualidade destes resultados em comparação ao cenário atual da empresa.