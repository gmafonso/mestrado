O desenvolvimento de \sw tem se tornado uma das atividades mais importantes no cen�rio tecnol�gico mundial, pois aplicativos e sistemas permeiam todas as atividades econ�micas, alcan�ando, ou no m�nimo afetando, todas as esferas sociais. As pequenas e m�dias empresas de desenvolvimento de \sw possuem grande papel neste cen�rio e representam grande parte desta ind�stria. Entretanto, seus processos n�o s�o bem formalizados, o que traz consequ�ncias negativas como atrasos nas entregas, estouro de or�amento e conflitos com o cliente. A ado��o de boas pr�ticas em ger�ncia de projetos e desenvolvimento de \sw atrav�s da implanta��o de normas t�cnicas como a \iso, criada especificamente para pequenos e m�dios desenvolvedores de \sw, � o caminho ideal para a profissionaliza��o da ind�stria. Por sua vez, o processo de ado��o desta ou de qualquer outra norma t�cnica possui um custo muito alto, em termos de tempo e recursos financeiros e humanos, dificultando sua implanta��o. Esta disserta��o tem como objetivo criar um processo de auto diagn�stico que permita a avalia��o do grau de maturidade de uma empresa e permita selecionar quais ser�o os primeiros passos na implanta��o da \iso, considerando os valores individuais de cada empresa para que os resultados observ�veis a curto prazo estejam alinhados com esses valores e se tornem mais atrativos, incentivando equipe e diretores, e impedindo uma desist�ncia prematura da implanta��o das melhorias.

Palavras chave: \iso, normas t�cnicas, desenvolvimento de \sw.