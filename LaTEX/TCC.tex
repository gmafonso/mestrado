\documentclass[12pt,a4paper]{fase}
\usepackage[portuges,brazil]{babel}
% \usepackage[portuguese,brazil]{babel}

%\usepackage[brazil]{babel}
\usepackage[latin1]{inputenc}
\usepackage{amssymb,amsmath}
\usepackage{epsfig}
\usepackage{multirow}
\usepackage{xspace}
\usepackage{url}
\usepackage{etoolbox}

\usepackage{isolatin1}
\usepackage{amssymb}
\usepackage{subfigure}
\usepackage{graphicx}
\usepackage{caption2}
\usepackage{setspace}
\usepackage{ps-macros}
% \usepackage{psfig}
\usepackage{natbib} % support the \includegraphics command and options
\usepackage{array}
\usepackage{floatrow}
\floatsetup[table]{capposition=top}
\usepackage{soul}
\setcounter{secnumdepth}{3}    % n - numero de niveis de subsubsection numeradas
\setcounter{tocdepth}{3}       % coloca ate o nivel n no sumario

%\renewcommand{\chapter}{\section}
\newcommand{\capitulo}[1]{\chapter{#1}}
\newcommand{\secao}[1]{\section{#1}}


\title{T\'itulo da Monografia}
\author{Nome do autor}
\advisortitle{Orientador}
\advisorname{Nome do Orientador}
\advisorplace{}
\date{01 de abril de 2016}
\city{Petr\'opolis}
\year{2016}

\banca        % nao insira o nome do orientador, ja eh feito automaticamente
{Prof. Ms/Dr}{}
{Prof. Ms/Dr}{}
{}{} % se nao houver deixe em branco {}{}
{}{}    % se houver um quarto membro na banca, inserir nome e instituicao

\defesa{ 2016 } % dia em que foi realizada a defesa da dissertacao

\newtoggle{full}
\toggletrue{full}
%\togglefalse{full}


\begin{document}


\makecapaproposta             % cria capa para proposta%

\makecapadissertacao           % cria capa para dissertacao de mestrado %
\makerosto                     % cria folha de rosto para versao final da FASE %
\maketermo                     % cria folha com o termo de aprovacao da dissertacao%

%\singlespacing           % espacamento 1 - capa FASE%
%\onehalfspacing          % espacamento 1/2 %
\doublespacing            % espacamento 2 - FASE %

\hyphenation{elements needed adequate verify}

\pagestyle{headings}
\pagenumbering{roman}



% !TEX root = TCC.tex
\null
\vfill
\begin{flushright}
	
DEDICAT�RIA
	
(opcional)
	
Ex: A meus pais por me terem dado a oportunidade de construir minha trajet�ria profissional

\end{flushright}	
	
\clearpage

Agrade�o primeiramente a Deus pela vida e por colocar as pessoas certas nas horas em que mais precisava para alcan�ar o sonho do Mestrado, � minha fam�lia, em especial � minha esposa M�nica e meus filhos Daniel e J�lia, ao meu orientador Prof. Dr. Ant�nio Roberto Mury, aos professores que se tornaram mestres e amigos, aos alunos e companheiros de caminhada que em muitas horas me auxiliaram e aos amigos de longa data que vieram a ser meus professores e incentivadores no in�cio desta caminhada: Prof. Dr. Ant�nio Tadeu Azevedo Gomes e Prof. Dr. F�bio Lopes Licht.
% !TEX root = TCC.tex
\null
\vfill
\begin{flushright}
	
	\textit{EP�GRAFE}
	
	(opcional)
	
	\textit{Ex: ``Para transformar � preciso colocar um p� no sonho,	outro na realidade e um bocado de ousadia louca..''}
	
	\textbf{\textit{Paulo Freire}}
	
\end{flushright}
\clearpage


Resumo. \\
Palavras chave: ISO 29110	.

Abstract.
Keywords: ISO 29110.

\tableofcontents


\listoffigures

\addcontentsline{toc}{chapter}{\MakeUppercase{Lista de Figuras}}
\newpage

\listoftables
\addcontentsline{toc}{chapter}{\MakeUppercase{Lista de Tabelas}}
\newpage


\pagenumbering{arabic}

% !TEX root = dissertacao.tex
\chapter{Introdu��o}
\label{Introducao}

(A introdu��o deve incluir o \textbf{tema} e breves coloca��es iniciais, apresentando as \textbf{motiva��es pessoais} do autor em rela��o ao tema)

\section{Apresenta��o do tema}
\label{intro.apres}

(Neste item, o \textbf{tema} deve ser apresentado, de forma geral e panor�mica, e progressivamente problematizado de modo a focar um objeto/problema de investiga��o.
O texto deve ser iniciado pela descri��o/caracteriza��o do tema, seguida de sua \textbf{contextualiza��o na realidade e no campo de conhecimento} onde se insere. O autor deve apresentar ideias claras e uma vis�o de conjunto, demonstrando que domina minimamente o assunto que pretende abordar. Deve mostrar diferentes �ngulos da quest�o, pois � o confronto de diferentes perspectivas que vai permitir a sua problematiza��o. 

A problematiza��o deve ser iniciada com a exposi��o das inquieta��es, d�vidas te�ricas e/ou das dificuldades pr�ticas que fomentaram a busca pela compreens�o do fen�meno e culminar com a formula��o do problema de pesquisa. 

O \textbf{problema espec�fico} a ser investigado ao longo do estudo, deve ser destacado no final do texto, de prefer�ncia na forma de pergunta e em negrito. Deve conter o \textbf{objeto} (``O que'' exatamente voc� pretende pesquisar?) e o \textbf{universo da pesquisa} (indiv�duo, grupo e institui��o pesquisada - quem? onde?).

\section{Objetivos}

\subsection{Geral}

No objetivo geral o autor apresenta qual o prop�sito de sua pesquisa, o que se pretende com o trabalho. Ele deve indicar a finalidade da pesquisa, relativa ao objeto e/ou a contribui��o que o estudo pretende dar ao campo de conhecimento e/ou aos profissionais que atuam na �rea. O \textbf{objetivo geral} do estudo refere-se especialmente ao conhecimento que se pretende construir a partir do projeto de investiga��o. Utilizar verbo no infinitivo: analisar, investigar, descrever, pesquisar, verificar, identificar, mapear, levantar, diagnosticar, avaliar, conceituar, relacionar, etc. ? consultar listagem entregue.

\subsection{Espec�ficos}

Os objetivos espec�ficos est�o voltados para as quest�es secund�rias a serem respondidas no decorrer da pesquisa e que s�o relacionadas � quest�o principal. S�o objetivos que v�o auxiliar o autor a atingir seu objetivo geral e, em geral, correspondem a \textbf{etapas espec�ficas de investiga��o}. (Que objetivos menores pretende alcan�ar para favorecer o objetivo geral?).

A reda��o, em t�picos, deve ser clara e objetiva e iniciar-se sempre com verbos no infinitivo como, por exemplo: analisar, investigar, descrever, pesquisar, verificar, identificar, etc.? consultar listagem entregue.

\subsection{Justificativa}

(Argumentar sobre a relev�ncia do estudo, o porque a realiza��o do seu trabalho � importante. Deve-se explicitar qual ou quais \textbf{ustificativas} existem para o desenvolvimento do estudo da problem�tica escolhida. 

Pode-se iniciar considerando a \textbf{\ul{abrang�ncia e seu impacto social, pol�tico e econ�mico do problema}}. Apresente dados que comprovem a \textbf{magnitude ou a gravidade do problema}. 
Al�m disso, � importante \ul{\textbf{situar a problem�tica no campo da administra��o}}, de modo que o leitor n�o tenha d�vidas de que este � um problema pertinente aos estudos nessa �rea ? n�o � apenas um problema com grande relev�ncia para a sociedade em geral, mas em especial, para os administradores. Indique outras pesquisas ou a an�lise de especialistas que apontem para como a \ul{\textbf{problem�tica escolhida tem gerado preju�zos, desafios �s organiza��es}}. 
Ap�s justificar o ?porque? o tema merece ser estudado, � preciso defender a \ul{\textbf{relev�ncia cient�fica}} que pode apresentar o estudo em quest�o, \ul{\textbf{destacando como os resultados da pesquisa podem ajudar os estudiosos a compreender melhor o problema, assim como os administradores a lidar com ele}}. Destaque a falta de estudos como esse no campo, a necessidade de investiga��es emp�ricas como essa, \ul{\textbf{as contribui��es que as an�lises, revis�es propostas, podem dar para a compreens�o do problema}} e o desenvolvimento cient�fico no campo de conhecimento espec�fico onde se insere assim como \ul{\textbf{para a melhoria os processos e pr�ticas de gest�o nas organiza��es}}. 

\subsection{Procedimentos metodol�gicos}

Refere-se a descri��o dos procedimentos intelectuais e t�cnicos da pesquisa. Deve esclarecer qual \textbf{o tipo de pesquisa utilizada e as op��es metodol�gicas para execu��o da mesma}. N�o deixe de justificar e fundamentar suas escolhas.

Nos estudos te�ricos/bibliogr�ficos, descrever os crit�rios que orientaram a op��es te�ricas feitas (o por qu� estar trabalhando com esse ou tal autor/teoria), as etapas de leitura e de documenta��o.
No caso de realiza��o de estudo de campo, � importante situar o leitor do campo escolhido para realiza��o da pesquisa. Indique \ul{\textbf{qual/quais institui��o/coes onde o trabalho de campo se realizou e quais foram os crit�rios/motivos para escolha da institui��o}}. Descreva a/as institui��o/�es de maneira objetiva: sua localiza��o, �rea de atua��o, dimens�es, estrutura f�sica e humana. 

� preciso ainda explicar com quem foram feitas as entrevistas/aplicados question�rios e por qu� (como se deu a sele��o dos participantes, os crit�rios de inclus�o e de exclus�o); e \ul{\textbf{caracterizar os sujeitos do estudo}} (sexo, idade, escolaridade, cargos/fun��es, etc).
A seguir, procure \ul{\textbf{detalhar as etapas de investiga��o}} de modo a que o leitor possa reconhecer todo o caminho percorrido: como foi a aproxima��o a institui��o/aos sujeitos, \ul{\textbf{quais os instrumentos de coleta de dados utilizados e o que voc� pretendia com cada instrumento}}.
 
� necess�rio mencionar o tipo de entrevista escolhida (estruturada, semi-estruturada, n�o-estruturada, grupo focal) e o tipo de question�rio escolhido (com perguntas abertas, fechadas, escala de valores, etc), justificando suas escolhas. Tanto o roteiro de entrevistas quanto o modelo do question�rio n�o devem ser detalhados nesse subitem, mas podem estar integralmente no ap�ndice do projeto. 

� preciso ainda \ul{\textbf{apresentar os procedimentos para a an�lise dos dados}} (categoriza��o, an�lise de conte�do/discurso, tratamento estat�stico, etc).
Importante: toda a metodologia da pesquisa deve estar em conson�ncia com o problema/objetivos do estudo e o referencial te�rico adotado. 

% !TEX root = dissertacao.tex
\chapter{(T�TULO DO CAP�TULO)}
\label{cap.2}

Neste cap�tulo dever� ser apresentada a fundamenta��o te�rica da monografia, podendo ser subdividido em subitens se necess�rio. (Utiliza��o m�nima de cinco fontes)
% !TEX root = dissertacao.tex
\chapter{Descri��o da Proposta}
\label{proposta}

Como visto no Cap�tulo \ref{Sec:intro:problema}, era necess�rio encontrar uma solu��o que tornasse vi�vel a implanta��o da norma \iso em uma pequena empresa de desenvolvimento de \sw com o menor risco poss�vel de paralisa��o e abandono nas suas fases iniciais.

Muitas empresas e especialistas em implanta��o da norma \iso se utilizam de question�rios para realizar a avalia��o inicial do estado atual das empresas de \sw e determinar o caminho que dever� ser percorrido, atrav�s da implanta��o dos processos da norma, para se alcan�ar o estado desejado. Por�m, estes question�rios n�o s�o disponibilizados para um auto diagn�stico e � necess�rio arcar com o alto custo da consultoria inicial para conseguir os resultados que podem demorar a ser divulgados. Ademais, os question�rios de avalia��o colocam todas as atividades e tarefas da \iso em um mesmo patamar de import�ncia, sem levar em considera��o os valores das empresas alvo. 

A proposta desta disserta��o � a cria��o de um question�rio de auto diagn�stico para implanta��o da \iso que priorize processos com maior ader�ncia aos valores individuais de cada empresa e que tragam benef�cios relevantes observ�veis nas etapas iniciais da implanta��o.

\section{Defini��o do Problema}
\label{sec:def:prob}

O primeiro passo tomado neste estudo foi formalizar a defini��o do problema e a sua delimita��o. Foi constatado que nenhum m�todo preexistente era capaz de auxiliar as empresas conforme a proposta colocada anteriormente. Para ent�o procurar cobrir essa lacuna o seguinte problema para estudo foi definido:

\begin{itemize}

\item Pequenas empresas de \sw geralmente possuem grandes limita��es de recursos financeiros, humanos e materiais para executar projetos de melhorias de processos, principalmente os que representam custos maiores como normas ISO. Mesmo cientes dos benef�cios que podem representar, muitos empres�rios se mostram receosos em implantar essas solu��es por conta dos altos riscos de insucesso provenientes da desmotiva��o que se abate nos est�gios iniciais onde o trabalho � muito dispendioso e os benef�cios observ�veis s�o pequenos ou nulos.

\item A an�lise inicial que determina o estado atual da empresa desenvolvedora de \sw � realizado atrav�s de uma empresa ou consultor especializado, cujo custo pode ser muito alto, a metodologia n�o � acess�vel e os resultados podem demorar a chegar nas m�os dos clientes.

\item As a��es de corre��o e melhoria sugeridas a partir da an�lise inicial n�o levam em considera��o os valores da empresa e colocam no mesmo patamar todas as atividades e tarefas. Ao implantar uma a��o que n�o traga um benef�cio relevante, os membros da equipe podem se sentir desmotivados e o projeto fica mais sujeito � paralisa��es e um poss�vel abandono.

\end{itemize}


\section{Justificativa do Trabalho}

Este trabalho tem como justificativa principal a demanda por melhorias j� existente na empresa foco. Conforme an�lise realizada no Cap�tulo \ref{Sec:analise:org}, � de import�ncia estrat�gica para a empresa que as melhorias sejam implantadas para suportar o crescimento vigente, minimizando os poss�veis riscos de paralisa��o e abandono do projeto, fatos que j� ocorreram no passado.

Al�m da aplica��o imediata na empresa foco, a ferramenta de auto diagn�stico criada a partir desta disserta��o tamb�m servir� de apoio a outras empresas de desenvolvimento de \sw que pretendam implantar a norma \iso ou a empresas e consultores especializados que poder�o se beneficiar de uma an�lise inicial mais focada nos valores das empresas.

\section{Metodologia e Desenvolvimento}

O trabalho foi realizado em tr�s etapas distintas: cria��o do question�rio de auto avalia��o, aplica��o do question�rio e an�lise dos resultados obtidos.

\subsection{Cria��o do question�rio de auto avalia��o}

A fim de produzir o question�rio de auto avalia��o, o autor pesquisou na literatura outros autores que houvessem passado por um problema semelhante e tivessem produzido uma ferramenta com as mesmas caracter�sticas. N�o foi poss�vel encontrar trabalhos com a mesma finalidade, mas foi obtivemos sucesso em coletar algumas informa��es sobre m�todos de cria��o de pesquisas de mercado, que se utilizam de question�rios como ferramenta principal de aplica��o.

Dentre os trabalhos encontrados, o artigo de \cite{manzato} define uma abordagem estat�stica para pesquisas qualitativas, que foi adaptada �s necessidades desta disserta��o e representada na Figura \ref{Fig:abordagem:pesquisa}. A seguir ser�o analisados os elementos desta abordagem dentro desta disserta��o.

\begin{figure}[!h]
	\centering
	\includegraphics[scale=0.5]{figuras/abordagem_pesquisa.png}
	\caption{Abordagem estat�stica na pesquisa quantitativa\newline Fonte: \cite{manzato}}
	\label{Fig:abordagem:pesquisa}
\end{figure}

\subsubsection{Problema}


O problema foi muito bem delimitado nesta disserta��o, conforme a Se��o \ref{Sec:intro:problema}.

\subsubsection{Planejamento amostral}

Por se tratar de um question�rio individual e o p�blico alvo ser bem estratificado, composto de pequenas empresas de desenvolvimento de \sw, n�o houve necessidade de planejamento amostral.

\subsubsection{Planejamento do question�rio}

Este � o ponto do processo onde esta disserta��o se diferencia dos demais trabalhos, pois seria necess�rio planejar n�o somente as quest�es que iriam compor o question�rio, mas tamb�m de que forma elas seriam impactadas pelo peso de cada valor empresarial. Para atingir este objetivo era necess�rio planejar a classifica��o dos valores empresariais antes mesmo das perguntas do question�rio.

Outro fator muito importante neste ponto do processo era a necessidade de criar um question�rio pr�tico, simples de ser respondido e cujas respostas fossem geradas de forma imediata e sem necessidade de intera��o com terceiros (especialistas, por exemplo). Para tanto, o autor tomou a decis�o de criar um question�rio digital que, em um primeiro momento foi confeccionado em uma planilha eletr�nica, podendo ser, posteriormente, transposto para um \sw ou p�gina de internet de forma simples e mantendo sua l�gica de funcionamento.

O primeiro passo foi a cria��o da classifica��o dos valores empresariais, cujo resultado pode ser observado na Figura \ref{Fig:class:valores}. Podemos notar tr�s estruturas principais: os valores empresariais na primeira coluna, as marca��es da classifica��o de import�ncia nas colunas seguintes e o c�lculo do peso final na �ltima coluna. Para simplificar a marca��o de import�ncia dos valores, foi colocado um guia visual na primeira linha que indica que as marca��es aumentam de valor da esquerda para a direita.

\begin{figure}[!h]
	\centering
	\includegraphics[scale=0.7]{figuras/class_valores.png}
	\caption{Classifica��o dos valores empresariais}
	\label{Fig:class:valores}
\end{figure}



\subsection{title}

\subsubsection{Defini��o do problema a ser investigado}

Este passo j� havia sido estipulado desde o in�cio desta disserta��o, onde foi detectado o problema a ser tratado pela mesma e descrito em \ref{Sec:intro:problema}.

\subsubsection{Defini��o dos objetivos da pesquisa}

Os objetivos que subsidiam a elabora��o do question�rio s�o:

\begin{itemize}
	
	\item Coleta de dados sobre a situa��o atual da empresa em rela��o aos processos da \iso;
	
	\item Categoriza��o dos resultados de acordo com valores empresariais mais relevantes;
	
	\item Gera��o de uma lista de melhorias sugeridas ordenadas pelos valores empresariais.
	
\end{itemize}

\subsubsection{Defini��o do p�blico alvo da pesquisa}

Em um primeiro momento o p�blico alvo da pesquisa era somente a empresa alvo da disserta��o. Por�m, o question�rio foi pensado para atender as organiza��es do mesmo segmento: pequenas empresas desenvolvedoras de \sw, chamadas de VSE.

\subsubsection{Defini��o da t�cnica a ser utilizada}

Por se tratar de um question�rio de auto avalia��o, a t�cnica a ser utilizada ser� a entrevista individual conduzida pelo pr�prio pesquisado.

\subsubsection{Sele��o da amostra}

No primeiro momento somente a empresa alvo da disserta��o foi utilizada na pesquisa.

\subsection{Aplica��o do question�rio}

Ainda seguindo o roteiro de elabora��o e aplica��o de uma pesquisa adaptado de \cite{manzato}, os passos seguintes foram a realiza��o e transcri��o da entrevista individual.

O primeiro passo para a realiza��o da pesquisa era atribuir pesos aos valores empresariais definidos na elabora��o do question�rio. Por se tratar de uma classifica��o visual, onde era poss�vel enxergar  os pesos visualmente em uma escala 

O autor preencheu a planilha do question�rio, respondendo as quest�es de forma que refletissem a realidade atual da empresa. O processo, apesar de ser longo pela grande quantidade de perguntas, se mostrou muito fluido e simples.

\subsection{An�lise dos resultados obtidos}


O primeiro passo para a constru��o de um question�rio de auto avalia��o de implanta��o da norma \iso, ou de qualquer outro tipo de question�rio, � o planejamento das quest�es que ser�o aplicadas. .

\begin{figure}[!h]
	\centering
	\includegraphics[scale=0.8]{figuras/fluxo_plan_questionario.jpg}
	\caption{Roteiro para elabora��o e aplica��o de uma pesquisa\newline (adaptado de \cite{manzato}}
	\label{Fig:fluxo:plan:quest}
\end{figure}


\subsection{Revis�o bibliogr�fica da \iso}

\subsection{Projeto}

O projeto foi a etapa onde as a��es necess�rias para a melhoria dos processos foram elencadas, priorizadas e devidamente documentadas. Tamb�m foram identificados os principais atores (\textit{stakeholders}), o cronograma, os recursos necess�rios e outros elementos, conforme diretrizes do PMBOK\footnotemark{} criado pelo PMI\footnotemark. Nesta etapa foram desenvolvidas as ferramentas de \sw que buscam facilitar as melhorias atrav�s da integra��o de informa��es vitais para os processos identificados.

\footnotetext{\textit{Project Management Book Of Knowledge}, guia de boas pr�ticas de ger�ncia de projetos mundialmente reconhecido}
\footnotetext{\textit{Project Management Institute}, institui��o internacional refer�ncia em ger�ncia de projetos.}

\subsection{Conclus�o}

A conclus�o consistiu na coleta dos resultados obtidos ap�s a implanta��o do projeto de melhoria dos processos e an�lise da qualidade destes resultados em compara��o ao cen�rio atual da empresa.

\section{Estudo de caso}

\subsection{An�lise Organizacional e de Processos}
\label{Sec:analise:org}


\textbf{Diagn�stico:} observou-se que a empresa possui qualidades essenciais para o crescimento cont�nuo, como por exemplo, o comprometimento dos profissionais, a comunica��o, o bom clima organizacional, assim como a cultura de prezar pela excel�ncia e ser reconhecida atrav�s da sua confiabilidade e qualidade nos servi�os. Por�m, para que a empresa suporte o crescimento que tende a acontecer cada vez mais, devido a demanda pelos servi�os, torna-se necess�rio alguns reajustes nos processos.

\subsubsection{Suporte T�cnico}
\label{Sec:suporte}

Este departamento � o �nico respons�vel pelo atendimento ao cliente atualmente. Assuntos sobre problemas e d�vidas relacionados aos \sws fornecidos s�o tratados diretamento com os t�cnicos do suporte. Para isso o departamento se utiliza de 2 \sws, cujas especifica��es est�o relacionadas na Tabela \ref{Tab:espec:sw:atend} e as janelas principais podem ser observadas nas Figuras \ref{Fig:jan:contatos:lanca} e \ref{Fig:jan:pendencias:lanca}.

\begin{table}[h!]\footnotesize
	\centering
	\begin{tabular}
		{
			|p{1,5cm}|p{12cm}|
		}
		
		\hline
		\textbf{Nome}&
		\textbf{Especifica��es}\\
		\hline
		
		Contatos&
		\begin{itemize}
			\item Registra atendimentos;
			\item Registra o cliente que est� sendo atendido;
			\item Possui um espa�o virtualmente ilimitado para informar o assunto do contato;
			\item Cronometra automaticamente o atendimento;
			\item Possui uma data de previs�o de retorno;
			\item N�o permite inclus�es de novas intera��es com o cliente durante o andamento do atendimento, que pode se estender por dias (neste caso novos atendimentos dever�o ser registrados, sem haver qualquer liga��o entre a primeira e as demais chamadas).
		\end{itemize}\\
		\hline
		
		Pend�ncias&
		\begin{itemize}
			\item Registra tarefas;
			\item Registra o cliente que fez a solicita��o (opcional);
			\item Associa o solicitante e o respons�vel por sua realiza��o (ambos colaboradores da empresa);
			\item Permite definir prioridade, data limite para realiza��o e previs�o de atendimento;
			\item Possui um controle de fechamento, onde o respons�vel indica a data em que aquela tarefa foi realizada;
			\item Possui um controle de ``ok'', onde o solicitante indica que aquela tarefa j� foi liberada para o cliente;
			\item Diferentemente do Contatos, permite acrescentar novas tarefas ao mesmo registro principal.
		\end{itemize}\\
		\hline
		
	\end{tabular}
	\caption {Especifica��es dos \sws de atendimento}
	\label{Tab:espec:sw:atend}
\end{table}

\begin{figure}[!h]
	\centering
	\includegraphics[scale=0.6]{figuras/jan_contatos_lancamento.jpg}
	\caption{Janela de lan�amento dos atendimentos}
	\label{Fig:jan:contatos:lanca}
\end{figure}

\begin{figure}[!h]
	\centering
	\includegraphics[scale=0.6]{figuras/jan_pendencias_lancamento.jpg}
	\caption{Janela de lan�amento das pendencias}
	\label{Fig:jan:pendencias:lanca}
\end{figure}

Os atendimentos podem ser realizados atrav�s do telefone, acesso remoto via internet, e-mail ou presencial. Como pode ser observado na Figura \ref{Fig:atend11}, somente 3 tipos foram registrados no m�s de novembro de 2014, mostrando que as comunicac�es via e-mail n�o s�o registradas.

\begin{figure}[!h]
	\centering
	\includegraphics[scale=0.5]{figuras/atendimentos_por_tipo.jpg}
	\caption{Atendimentos por tipo em novembro/2014}
	\label{Fig:atend11}
\end{figure}

Outros problemas foram diagnosticados e est�o relacionados na Tabela \ref{Tab:probl:atend}, sendo a falta de acompanhamento do andamento das solicita��es e retorno ao cliente as mais cr�ticas.

\begin{table}[h!]\footnotesize
	\centering
	\begin{tabular}
		{
			|p{14cm}|
		}
		
		\hline
		\textbf{Problema diagnosticado}\\
		\hline
		
		Falta de registro de atendimento (para qualquer tipo)\\
		\hline
		
		Posterga��o de registro de atendimento (para qualquer tipo), que pode levar ao esquecimento (falta de registro)\\
		\hline
		
		Dados insuficientes sobre o contato\\
		\hline
		
		Falta de acompanhamento do andamento das solicita��es (fechamento dos registros de atendimento)\\
		\hline
		
		Falta de retorno da situa��o das solicita��es ao cliente\\
		\hline
		
		Tarefas originadas dos atendimentos, para o pr�prio departamento de suporte ou para outros departamentos, s�o registradas em um \sw separado e n�o h� nenhuma rastreabilidade\\
		\hline
		
	\end{tabular}
	\caption {Problemas diagnosticados no departamento de suporte}
	\label{Tab:probl:atend}
\end{table}

As seguintes a��es devem ser tomadas para contonar os problemas diagnosticados:

\begin{itemize}
	
	\item Todos os chamados t�cnicos dever�o ser registrados no ato de sua execu��o (liga��o, acesso remoto ou recebimento do e-mail), com exce��o do atendimento externo que deve ser registrado no ato do retorno do t�cnico;
	
	\item O m�ximo de informa��es poss�veis precisa ser registrado no hist�rico do chamado;
	
	\item A fus�o entre o Contatos e o Pend�ncias � extremamente necess�ria, a fim de criar um registro de todas as a��es vinculadas naquele chamado, tornando-se obrigat�rio o registro de todas as a��es realizadas por cada t�cnico que atender o chamado;
	
	\item Qualquer nova informa��o de um chamado deve ser adicionada ao registro j� aberto, sem a necessidade de abrir um novo chamado e permitindo a rastreabilidade;
	
	\item Um departamento de Servi�o de Atendimento ao Cliente (SAC) dever� ser criado e pelo menos uma pessoa deve ser colocada como respons�vel pelas suas atribui��es;
	
	\item O SAC realizar� o fechamento do chamado junto ao cliente, consultando se realmente o problema foi solucionado e realizando pesquisas de satisfa��o e gerando indicadores de desempenho e qualidade (tempo m�dio de conclus�o dos chamados, problemas mais ocorridos, cliente mais ativo, t�cnico mais ativo, etc);
	
	\item Os clientes dever�o receber um Documento de Abertura e Acompanhamento de Chamados, juntamente com o manual  do \sw, para que saibam exatamente como proceder para abrir e acompanhar um chamado;
	
	\item Para uma melhor performance da equipe de suporte, dever� ser feito a gest�o do conhecimento, documentando a resolu��o de cada problema que surja no atendimento do suporte t�cnico;
	
	\item � necess�ria a elabora��o de um fluxograma do atendimento, com as informa��es chaves de requisitos para abertura de chamados;
	
	\item � necess�ria a elabora��o de um fluxograma do atendimento de servi�os diferenciados, tais como treinamentos, suportes avulsos, entrada de equipamento para concerto ou manuten��o, entre outros, lembrando de acrescentar no fluxo a emiss�o da Ordem de Servi�o;
	
	\item Dever� ser realizado o acompanhamento dos clientes que n�o abrem chamado com o suporte h� mais de 3 meses.
	
\end{itemize}

\subsubsection{Desenvolvimento}
\label{Sec:desenvolvimento}

Este departamento n�o tem contato direto com os clientes, pois todas as solicita��es passam pelo departamento de suporte t�cnico. Por�m, todas as solcita��es de mudan�a ou corre��o de problemas nos \sws s�o resolvidas por este departamento e alguns atendimentos s�o repassados para o setor de desenvolvimento para resolu��o conjunta quando os t�cnicos n�o possuem conhecimento ou capacidade para trat�-los por si mesmos. 

As tarefas deste departamento s�o registradas no \sw Pend�ncias mas, como j� relatado anteriormente, n�o possuem relacionamento com os chamados abertos no \sw Contatos. Todos os problemas diagnosticados est�o relacionados na Tabela \ref{Tab:probl:desenv}.

\begin{table}[h!]\footnotesize
	\centering
	\begin{tabular}
		{
			|p{14cm}|
		}
		
		\hline
		\textbf{Problema diagnosticado}\\
		\hline
		
		Falta de posicionamento quanto ao andamento das solicita��es\\
		\hline
		
		Falta de previs�o de entrega das solu��es\\
		\hline
		
		Falta de rastreabilidade entre abertura de chamados e tarefas\\
		\hline
		
	\end{tabular}
	\caption {Problemas diagnosticados no departamento de desenvolvimento}
	\label{Tab:probl:desenv}
\end{table}

Para solucionar os problemas diagnosticados para este setor, al�m das a��es ja citadas em \ref{Sec:suporte}, ser�o necess�rias as seguintes a��es:

\begin{itemize}
	
	\item Criar processos para determinar a previs�o de lan�amento de vers�es de \sws;
	
	\item Criar processos para determinar em qual vers�o determinada solicita��o ser� inclu�da;
	
	\item Integrar aos \sws de atendimento as informa��es de lan�amento de vers�es e, consequentemente, a previs�o das solicitac�es.
	
\end{itemize}

\subsubsection{Financeiro}

O departamento financeiro lida com o cliente com uma frequ�ncia menor que os departamentos de suporte e desenvolvimento. Por�m, os assuntos relacionados a este departamento podem gerar transtornos e preju�zos quando feitos de forma incorreta. Al�m disso, este departamento tamb�m � respons�vel por bloquear o atendimento aos clientes inadimplentes, portanto representa um papel importante nos processos descritos em \ref{Sec:suporte} e \ref{Sec:desenvolvimento}.

Nenhum atendimento ou tarefa relacionados a este departamento s�o registrados nos \sws de atendimento. Portanto, as a��es necess�rias para melhoria dos processos s�o:

\begin{itemize}
	
	\item Registrar todo e qualquer contato com clientes nos \sws de atendimento;
	
	\item Integrar as informa��es financeiras com os \sws de atendimento para que a libera��o ou bloqueio seja feito automaticamente no ato da identifica��o do cliente.
	
\end{itemize}

\subsubsection{Marketing}

O departamento de marketing � o respons�vel, geralmente, pelo primeiro contato com o cliente. Durante as negocia��es de venda de \sw, � comum haver solicita��es de modifica��es ou acertos sobre configura��es, convers�es de dados e treinamentos. Por�m, nenhuma dessas informa��es s�o registradas nos \sws de atendimento. O mesmo ocorre na venda de equipamentos e outros servi�os.

Portanto, as a��es necess�rias para melhoria dos processos s�o:

\begin{itemize}
	
	\item Registrar os primeiros contatos com todos os prospectos, mesmo que n�o se tornem clientes;
	
	\item Registrar toda e qualquer intera��o com os clientes, novos ou antigos;
	
	\item Incluir nos registros as informa��es sobre negocia��o, poss�veis modifica��es, convers�es e outras condi��es estabelecidas durante ou ap�s a venda.
	
\end{itemize}


\subsection{An�lise SWOT}


A fim de resumir e melhor entender a an�lise organizacional e de processos realizada em \ref{Sec:analise:org}, foi desenvolvida a Tabela \ref{Tab:SWOT} com a an�lise SWOT.

%\begin{table}[h!]\footnotesize
%\centering
%\begin{tabular}
%{
%	| >{\centering\arraybackslash}p{7cm}
%	| >{\centering\arraybackslash}p{7cm}|
%}
%\hline
%
%	For�as&
%	Fraquezas\\
%	\hline
%
%	%For�as
%	\begin{tabular}{p{6,8cm}}
%	Comprometimento dos profissionais;\\
%	Bom clima organizacional;\\
%	Cultura de prezar pela excel�ncia;\\
%	Ser reconhecida atrav�s da sua confiabilidade e qualidade nos servi�os.\\
%	\end{tabular}&
%
%	%Fraquezas
%	\begin{tabular}{p{6,8cm}}
%	Falta de informa��es no acompanhamento de solicita��es;\\
%	Inefici�ncia na comunica��o com o cliente.\\
%	\end{tabular}\\
%	\hline
%
%	Oportunidades&
%	Amea�as\\
%	\hline
%
%	% Oportunidades
%	\begin{tabular}{p{6,8cm}}
%	Taxa de crescimento elevada nos �ltimos meses;\\
%	Aumento no faturamento;\\
%	Alta divulga��o da marca, dentro e fora da sua pr�pria cidade.\\
%	\end{tabular}&
%
%	% Amea�as
%	\begin{tabular}{p{6,8cm}}
%	Perder a confiabilidade dos clientes;\\
%	Manchar a reputa��o;\\
%	Sofrer queda nas vendas pela perda de indica��es de clientes e parceiros insatisfeitos;\\
%	Sofrer perda de clientes j� estabelecidos por n�o prestar um bom atendimento.\\
%	\end{tabular}\\
%	\hline
%
%\end{tabular}
%\caption {An�lise SWOT dos processos}
%\label{Tab:SWOT}
%\end{table}


\subsection{Estrat�gia}

\label{Sec:ec:estrategia}

A implanta��o de um novo processo, ou de sua melhoria, � uma atividade que possui um custo financeiro, de tempo e de recursos muito alto para qualquer empresa, independente do seu porte. A implanta��o de processos baseados em normas pode ser ainda mais custoso e complexo, principalmente para uma pequena empresa.

Por conta destes fatores, a empresa objeto de estudo desta disserta��o optou por realizar a implanta��o incremental da \iso, escolhendo as �reas mais deficientes citadas no diagn�stico que se encontra no Cap�tulo \ref{Cap:analise:cenario}.

Esta disserta��o teve como foco o tratamento dos problemas identificados no departamento de \textbf{Desenvolvimento}, descritos na Se��o \ref{Sec:desenvolvimento}. Apesar do foco ter sido em somente um departamento, foi observado que as solu��es propostas afetariam diretamente os demais departamentos, pois havia uma raiz comum a todos os problemas diagnosticados: a comunica��o com o cliente. A cria��o de processos bem estruturados de registro de atendimentos em conjunto com ferramentas que suportassem estes processos cobririam todos os problemas diagnosticados.

Outro fato de suma import�ncia observado foi a abrang�ncia da solu��o proposta. Apesar do estudo de caso ter sido feito em uma empresa de desenvolvimento de \sw, outros segmentos podem se beneficiar do processo de integra��o entre registro de contato com o cliente e controle de tarefas. 

Para se alcan�ar o objetivo desta disserta��o, foi adotada como estrat�gia inicial a sele��o dos objetivos da \iso, descritos nas Se��es \ref{Sec:iso:obj:gp} e \ref{Sec:iso:obj:dsw}, que pudessem contribuir para a solu��o dos problemas identificados na an�lise realizada no Cap�tulo \ref{Cap:analise:cenario}.

Ap�s a sele��o e an�lise dos objetivos da \iso, foram relacionadas as atividades mais importantes de cada um destes objetivos que dariam suporte ao alcance dos objetivos deste trabalho.

\subsubsection{Sele��o dos objetivos da \iso}
\label{Sec:estr:obj:iso}

Dentre os objetivos da \iso encontrados em \cite{iso}, foram selecionados 2 referentes � \gp e 1 referente ao \dsw. Podemos observar na Tabela \ref{Tab:estrat:obj:iso} a descri��o destes objetivos e a justificativa do porqu� foram selecionados.

\begin{table}[h!]\footnotesize
	\centering
	\begin{tabular}
		{
			|p{7cm}
			|p{7cm}|
		}
		
		\hline
		
		\textbf{Objetivo \iso}&
		\textbf{Problemas relacionados ao processo atual}\\
		\hline
		
		PM.03 A \muda � abordada atrav�s de sua recep��o e an�lise. Mudan�as aos requisitos de \sw s�o avaliadas em custo, cronograma e impacto t�cnico.&
		N�o existe processo formal para recep��o, an�lise e avalia��o da \muda.\\
		\hline
		
		%	PM.O4 Reuni�es de revis�o s�o realizadas com a equipe de trabalho e o cliente. Acertos s�o registrados e rastreados.&
		%	Reuni�es com o cliente n�o s�o registradas e os acertos s�o cadastrados no \sw Pend�ncias.\\
		%	\hline
		
		PM.O6 Uma \vcs de \sw � desenvolvida. Itens da \swcfg s�o identificados, definidos e inclu�dos em uma \bline. Modifica��es e entregas de um item s�o controladas e disponibilizadas ao cliente e equipe de trabalho. O armazenamento, manuseio e entrega dos itens s�o controlados.&
		Entregas de novas vers�es s�o realizadas sem nenhum controle de previs�o e aviso aos clientes.\\
		\hline
		
		SI.O6 Uma \swcfg, que cumpra com o \req acertado com o cliente, que inclua documenta��es de usu�rio, opera��o e manuten��o � integrada, inclu�da na \bline e armazenada no \rep. Necessidades de mudan�a na \swcfg s�o detectadas e os pedidos de mudan�a relacionados s�o iniciados.&
		N�o existe processo formal para recep��o, an�lise e avalia��o da \muda.\\
		\hline
		
	\end{tabular}
	\caption{Sele��o dos objetivos da \iso}
	\label{Tab:estrat:obj:iso}
\end{table}

\subsubsection{Sele��o das atividades dos objetivos da \iso}


Dentre as atividades que d�o suporte aos objetivos da \iso \cite{iso} selecionados em \ref{Sec:estr:obj:iso}, foram selecionadas as mais importantes e exequ�veis nesta primeira etapa de implanta��o dos processos na empresa objeto deste trabalho. As atividades selecionadas se encontram na Tabela \ref{Tab:estrat:ativ:iso}.

As atividades iniciadas com a sigla PM s�o relacionadas � \gp e as iniciadas com a sigla SI ao \dsw.

\begin{table}[h!]\footnotesize
	\centering
	\begin{tabular}
		{
			|p{7cm}
			|p{3,5cm}
			|p{3,5cm}|
		}
		
		\hline
		
		\textbf{Atividade \iso}&
		\textbf{Entradas}&
		\textbf{Sa�das}\\
		\hline
		
		PM.1.2 Definir com o Cliente as \entrega para cada um dos entreg�veis especificados na \dt.&
		\dt (revisada)&
		\ppj (\entrega)\\
		\hline
		
		PM.1.4 Estabelecer a dura��o estimada para realizar cada tarefa.&
		\ppj\par\begin{itemize}\item Tarefas\end{itemize}&
		\ppj\par\begin{itemize}\item Dura��o estimada\end{itemize}\\
		\hline
		
		PM.1.7 Atribuir datas estimadas de in�cio e t�rmino para cada uma das tarefas a fim de criar o \crono levando em considera��o os recursos, sequ�ncia e depend�ncias das tarefas.&
		\ppj\par\begin{itemize}
			\item Tarefas
			\item Dura��o estimada
			\item Composi��o da Equipe de Trabalho
		\end{itemize}&
		\ppj\par\begin{itemize}
			\item \crono
		\end{itemize}\\
		\hline
		
		PM.2.2 Analisar e avaliar a \muda em rela��o ao custo, cronograma e impacto t�cnico. A mudan�a solicitada pode ser iniciada pelo cliente ou pela equipe interna de trabalho. Atualize o \ppj se as mudan�as aceitas n�o afetam acordos com o cliente. Solicita��es de mudan�a que afetem esses acordos devem ser negociadas por ambas as partes.&
		\muda (iniciada)\par\ppj&
		\muda (avaliada)\par\ppj (atualizado)\\
		\hline
		
		PM.3.3 Identificar mudan�as aos requisitos e/ou \ppj relacionados � desvios significativos, riscos potenciais ou problemas relacionados � realiza��o do plano, documentando-os na \muda e acompanhando-os at� sua conclus�o.&
		\prog (avaliado)\par&
		\muda (iniciada)\\
		\hline
		
		%	PM.3.2 Estabelecer a��es para corrigir desvios ou problemas e identificar riscos relacionados � realiza��o do \ppj, conforme necess�rio, documentando-os no \corre e acompanhando-os at� o seu fechamento.&
		%	\prog (avaliado)\par&
		%	\corre\\
		%	\hline
		
		SI.6.6 Realizar entregas de acordo com as \entrega.&
		\ppj\par
		\begin{itemize}
			\item \entrega
		\end{itemize}\par
		\swcfg&
		\swcfg (entregue)\\
		\hline
		
	\end{tabular}
	\caption{Atividades da \iso}
	\label{Tab:estrat:ativ:iso}
\end{table}


%	Nesta se��o ser� apresentada a metodologia utilizada para a verifica��o das hip�teses apresentadas na Se��o \ref{hipoteses}.\\

%\subsection{Infraestrutura (Ambiente Real)}
%	Para execu��o dos testes foram criados e configurados 3 servidores com as seguintes especifica��es:
%	\begin{enumerate}
%		\item Processador Intel(R) Xeon(R) CPU X5650 2.67GHz (12 n�cleos)
%		\item 24 Gb de mem�ria RAM.
%		\item Disco R�gido SATA de 500 Gb.
%		\item Sistema Operacional Ubuntu Server 12.04.
%	\end{enumerate}
%	
%\subsection{Infraestrutura (Ambiente Virtual)}	
%	Em cada servidor real foram criados 3 servidores virtuais que foram instanciados simultaneamente para os testes. Estes servidores foram configurados da seguinte maneira:
%	\begin{enumerate}
%		\item QEMU Virtual CPU version 1.0 (Processador Intel(R) Xeon(R) CPU X5650 2.67GHz (12 n�cleos))
%		\item 4 Gb de mem�ria RAM.
%		\item Disco R�gido SATA de 10 Gb.
%		\item Sistema Operacional Ubuntu Server 12.04.
%	\end{enumerate}
%
%\subsection{Tipos de Testes de Concorr�ncia Executados}	
%	Todos os servidores compartilharam os recursos de processamento em seus 12 n�cleos para verificar a concorr�ncia das aplica��es. Os testes foram executados com as seguintes combina��es de ambientes:
%	
%	\begin{enumerate}
%		\item Ambiente Real;
%		\item Ambiente Virtual;
%		\item Ambiente Real X Ambiente Virtual;
%		\item Ambiente Virtual X Ambiente Real;
%		\item Ambiente Real X Ambiente Real;
%		\item Ambiente Virtual X Ambiente Virtual.
%	\end{enumerate}
%	
%	Para cada uma das combina��es acima foram executados 20 testes com os 4 algoritmos citados na Se��o \ref{dwarfs}, sendo computados os tempos de execu��o (individualmente e concorrentemente) e comparados os seus desempenhos. Os tempos de execu��o individuais foram utilizados como linha de base para a compara��o com os demais testes.\\
%

%\section{Abordagem do Problema}
%\label{abordagemDoProblema}
%	Para o planejamento dos testes tomou-se por base as conclus�es dos trabalhos apresentados na Se��o \ref{trabalhosrelacionados}. Entretanto como j� citado, estes trabalhos nos conduzem para a necessidade de obter informa��es sobre o efeito da concorr�ncia em ambientes compartilhados. Em toda a pesquisa realizada at� o momento, muito foi feito para avaliar o impacto de ambientes virtuais em servidores reais, mas nenhum destes estudos dedicou-se ao efeito da concorr�ncia causada pelas aplica��es executadas nestes ambientes, e a contribui��o do tipo de bibliotecas utilizadas na implementa��o destas aplica��es. \\
%	%quando � necess�rio o uso da concorr�ncia entre aplica��es. E at� o momento do encerramento do trabalho, n�o houve nenhuma abordagem que propusesse o uso do conceito de tipos de aplica��es para tratar este problema.
%	
%	E neste sentido, para que fosse poss�vel avaliar o efeito do compartilhamento foram testadas quatro tipos de aplica��es desenvolvidas com dois tipos de bibliotecas de programa��o paralela, OpenCL e OpenMP. As duas bibliotecas foram utilizadas por apresentarem boa documenta��o e por permitir testes de desempenho em diversas CPUs. Outro ponto importante � que esta abordagem permitiu o uso do Rod�nia, \textit{Benchmark} utilizado nos testes, que � apresentado na Se��o \ref{rodinia} e que possui implementa��o em OpenCL apresentado na Se��o \ref{opencl} e em OpenMP apresentado na Se��o \ref{openmp}.\\
%	
%	Os testes com as duas bibliotecas (OpenCL e OpenMP) se mostraram necess�rios devido �s diferen�as encontradas nos tempos de execu��o quando usadas em um mesmo algoritmo.\\
%	
%	Para os testes foram utilizados quatro \textit{Benchmarks} do Rod�nia que correspondem � tr�s tipos de classes. As classes de Dwarfs que foram escolhidas neste trabalho s�o: �lgebra Linear Densa (DLA), Grade Estruturada (SG) e Grafo Transversal (GT). A escolha por essas tr�s classes foi porque h� um grande n�mero de aplica��es cient�ficas em diversas �reas, como foi apresentado na Figura \ref{fig:dwarfs-areas} e em destaque na Figura \ref{fig:dwarfs-areas-destaque}. Al�m disso, o trabalho se concentra nestas tr�s classes porque abrangem o maior conjunto de tipos de aplica��es catalogadas pela abordagem dos Dwarfs.\\
%	
%	\begin{figure}
%		\centering\includegraphics[width=150mm]{Figuras/areasDwarfsDestaque.png}
%		\caption{Destaque das classes de Dwarfs usadas neste trabalho.}
%		\label{fig:dwarfs-areas-destaque}
%	\end{figure}
%
%\subsection{Classes e Algoritmos Escolhidos}
%	As seguintes classes de algoritmos foram escolhidos para testes de concorr�ncia:
%\begin{enumerate}
%	\item �lgebra Linear Densa � uma classe de Dwarf que envolve um conjunto de operadores matem�ticos realizados em valores escalares, vetores ou matrizes, quando a maioria dos elementos da matriz ou vetor s�o diferentes de zeros. Densa neste Dwarf refere-se � estrutura de dados aceita durante a computa��o. A intensidade aritm�tica do c�lculo operando os dados s�o de operadores de baixa intensidade (escalar por vetores, vetor-vetor, matriz-vetor , matriz-matriz, redu��o do vetor, vetor de digitaliza��o e produto escalar) que carregam um n�mero constante de opera��es aritm�ticas por elemento de dados. Tem uma raz�o elevada de opera��es matem�ticas para carga e um elevado grau de interdepend�ncia entre segmentos de dados. Eles s�o a base de solucionadores mais sofisticados, como LU de decomposi��o (LUD) ou Cholesky e apresentam alta intensidade aritm�tica \cite{Kaiser:2010}. Aplica��es classificados como DLA s�o relevantes atrav�s de uma variedade de dom�nios. Por exemplo, em ci�ncia dos materiais para a f�sica molecular e ci�ncia em nanoescala; em garantia de energia para a combust�o, fus�o e energia nuclear; na ci�ncia fundamental como astrof�sica e f�sica nuclear; no projeto de engenharia aerodin�mica. Algoritmos representativos desta classe s�o LUD, matriz transposta, matriz triangular, algoritmos de agrupamento, como K-means e Fluxo de cluster, e muitos outros. Os experimentos desta tese utilizou algoritmos LUD e Kmeans.\\
%	\begin{enumerate}
%		\item LUD � um algoritmo para calcular as solu��es de um conjunto de equa��es lineares que decomp�e a matriz como o produto de uma matriz triangular inferior e uma matriz triangular superior para conseguir uma forma triangular, que pode ser utilizada para resolver um sistema de equa��es lineares facilmente. A matriz $A \in$ $\mathbb{R}^{n \times n}$ tem uma fatora��o LU \emph{iff} todos os seus valores principais s�o n�o-zeros, ou seja, $det(A[1:k, 1:k]) \neq 0$ for $k = 1 : n-1$.\\
%		
%		\item Kmeans � um algoritmo de agrupamento amplamente utilizado na minera��o de dados, � um m�todo que particiona $n$ pontos que est�o em espa�o $d-$dimensional em $k$ aglomerados. O algoritmo semea $k$ aglomerados inicialmente no centro e determina para cada ponto de dados o seu centro mais pr�ximo, e ent�o recalcula os novos centros como o meio de seus pontos atribu�dos. Este processo de atribui��o de pontos de dados e reajustar centros � repetido at� que ele se estabilize.\\
%	\end{enumerate}
%	\item Grafo Transversal � um tipo de Dwarf que deve atravessar um n�mero de objetos em um gr�fico e examinar as caracter�sticas desses objetos. Um gr�fico ou uma rede s�o abstra��es intuitivas e �teis para a an�lise de dados relacionais, onde as entidades singulares s�o representadas como v�rtices, e as intera��es entre elas s�o retratadas como bordas. Aos v�rtices e �s bordas podem ser adicionalmente atribu�dos atributos com base na informa��o que encapsulam. Tais algoritmos normalmente envolvem uma quantidade significativa de mem�ria de acesso aleat�rio para pesquisas indiretas e pouca computa��o \cite{Kaiser:2010}. Dom�nios cient�ficos que incluem aplica��es importantes nesta classe s�o as de bioinform�tica (MUMmer), gr�ficos e pesquisa (BFS e B+Tree).\\
%	\begin{enumerate}
%		\item B+Tree � uma �rvore $n-$�ria muitas vezes com grande n�mero de filhos por n�. B+Tree consiste em uma raiz, n�s internos e folhas. A raiz pode ser uma folha ou um n� com dois ou mais filhos. O valor principal de uma B+Tree � no armazenamento de dados para recupera��o eficiente em um contexto de armazenamento orientados para o bloco. Isto se d� porque B+Tree tem alta ``fanout" (n�mero de apontadores para n�s filhos em um n�, tipicamente na ordem de 100 ou mais), o que reduz o n�mero de opera��es de E/S necess�rias para obter um elemento da �rvore. A ordem, ou fator de ramifica��o, b de uma B+Tree mede a capacidade de n�s (ou seja, o n�mero de n�s filhos) para n�s internos da �rvore. O n�mero real de filhos para um n�, referido aqui como m, � limitada para n�s internos de modo que $[b/2] \leq m \leq b$. N�s folha n�o tem filhos, mas s�o limitados, de modo que o n�mero de chaves deve ser de pelo menos $[b/2]$ e no m�ximo $b-1$. Na situa��o em que a B+Tree est� quase vazia, ela cont�m apenas um n�, que � um n� folha. A raiz � tamb�m a �nica folha, neste caso. A este n� � permitido ter uma chave, se necess�rio, e, no m�ximo, $b$ \\
%	\end{enumerate}
%	\item Grade Estruturada s�o algoritmos que servem para organizar dados em uma grade multidimensional regular, onde a computa��o se d� como uma s�rie de atualiza��es desta grade. Para cada atualiza��o da grade, todos os pontos s�o atualizados com valores a partir de uma pequena vizinhan�a em torno de cada ponto. A vizinhan�a est� normalmente impl�cita nos dados e determinada pelo algoritmo. Devido ao seu paralelismo inerente e calculo de natureza intensa, aplica��es de grade estruturados s�o geralmente uma boa op��o para as arquiteturas multi-core como GPU. Algoritmos de grade estruturada aparecem em muitos dom�nios cient�ficos que s�o citados a seguir com respectivo exemplo de aplica��o: imagens m�dicas (leuc�citos, Parede Cora��o e filtro de part�culas), simula��es de f�sica (\textit{HotSpot}), processamento de imagem (\textit{Speckle Reducing Anisotropic Diffusion}) e simula��es biol�gicas (mi�citos) \cite{Springer:2011}.\\
%	\begin{enumerate}
%		\item Speckle Reducing Anisotropic Diffusion (SRAD) � uma aplica��o de processamento de imagem para imagens de ultrassom e de radar. Ele reduz o ru�do de uma imagem dada, mantendo suas caracter�sticas importantes. Al�m disso, cada elemento da grade estruturada representa um pixel da imagem.\\
%	\end{enumerate}
%\end{enumerate}
%
%	
%\subsection{Ambiente de Testes}
%
%	O ambiente de testes necessita ser preciso afim de evitar que as avalia��es sejam prejudicadas. Para conseguir uma base funcional � necess�rio que o sistema seja ajustado seguindo algumas m�tricas de testes, tanto de hardware quanto de software e desta forma criou-se uma estrutura livre de testes tendenciosos.\\
%	
%	Para a an�lise de desempenho nos testes realizados foram colhidas 20 amostras, que se mostraram suficientes para gerar uma base de dados confi�vel. Este n�mero foi conseguido ap�s avaliar-se o intervalo de confian�a\footnote{C�lculo que apresenta o grau de confian�a de amostras estat�sticas.} das amostras. Neste caso foi avaliado que, desconsiderando aquelas com tempos constantes, as que mostravam varia��es de tempo de execu��o nas mesmas compara��es n�o eram significativas e ficavam em um intervalo pr�ximo com 10 execu��es. Assim, para se evitar conclus�es incertas, optou-se por testar 20 vezes cada combina��o de algoritmos, mesmo aquelas que possu�am tempos de execu��o constantes.\\
%	
%	Foram executados 4 tipos de algoritmos (B+Tree, Kmeans, LUD e SRAD) com combina��es de 2 tipos de bibliotecas (OpenMP e OpenCL) em ambientes reais e virtuais. Inicialmente foram testados os algoritmos em um ambiente livre de concorr�ncia para verificar os tempos de execu��o, afim de ter uma base de tempos para comparar a perda causada pela concorr�ncia em rela��o � n�o concorr�ncia. Comparou-se ent�o a concorr�ncia com todos os 4 algoritmos implementados usando a biblioteca OpenMP em ambientes reais e virtuais, seguido de todos os 4 algoritmos implementados em OpenCL em ambientes reais e virtuais (considerou-se estes testes como homog�neos) e por fim testou-se a concorr�ncia entre algoritmos implementados com as duas bibliotecas tamb�m executando a concorr�ncia em ambientes reais e virtuais (considerou-se estes testes como heterog�neos).\\
%	
%	Com todas as combina��es poss�veis e desconsiderando os testes iniciais de cada implementa��o, buscando a adequa��o dos testes, foram feitos 7120 testes, o que gerou uma base de conhecimento abrangente e confi�vel. Os resultados de todos estes testes s�o apresentados no Cap�tulo \ref{resultados}.\\
%%	Inicialmente foi feita uma an�lise gerencial de um ambiente na nuvem composto por recursos distintos em ambientes remotos, a fim de conseguir uma base de dados que possa ser utilizada pela Intelig�ncia Computacional (IC). Uma vez dispon�vel tal base, ser� poss�vel prever problemas em ambientes reais, j� que se poder� analisar que o evento que ocorreu anteriormente e que houve um problema associado a ele, ou mesmo que um conjunto de servidores/aplica��es n�o devem ser usados em conjunto, pois podem degradar o servi�o. De posse dessa an�lise, � poss�vel prever a ...

%\subsection{Defini��es e Metas dos Testes}

%		Este trabalho tem o foco principal no teste de desempenho, mediante avalia��o de tempo de execu��o quando h� concorr�ncia entre aplica��es para o mesmo recurso f�sico. Qualquer varia��o dos requisitos estipulados pelos testes deve ser tomada como uma n�o-conformidade e deve ser tratada desta forma. Este � o motivo desta se��o e fazem-se necess�rias as seguintes defini��es sobre testes:
%	
%	\begin{itemize}
%		\item Capacidade - � a carga de trabalho total que um sistema pode manipular sem violar os crit�rios de aceita��o.

%		\item Investiga��o - Busca o recolhimento de informa��es relacionadas com a velocidade, escalabilidade e/ou caracter�sticas de estabilidade do sistema. A avalia��o � frequentemente utilizada para provar ou refutar hip�teses sobre a causa de um ou mais problemas de desempenho observado.
%		
%		\item Lat�ncia - � uma medida da capacidade de resposta que representa o tempo que leva para completar a execu��o de um determinado processo. Pode-se tamb�m representar a soma de todas as lat�ncias de v�rios processos.
%		
%		\item M�tricas - S�o obtidas pela execu��o de testes de desempenho. As m�tricas normalmente s�o obtidos atrav�s de testes de desempenho, que incluem a utiliza��o do processador ou mem�ria ao longo de um determinado tempo.
%		
%		\item Desempenho - Refere-se a informa��es relativas ao tempo de resposta de uma aplica��o e/ou os n�veis de utiliza��o de recursos.
%		
%		\item Teste de Desempenho - � uma ``investiga��o" feita para determinar e/ou validar a velocidade, escalabilidade e/ou caracter�sticas de estabilidade do sistema. Este teste � o superconjunto que cont�m todas as outras subcategorias de testes de desempenho.


%		\item Teste de Unidade - � qualquer teste que visa as caracter�sticas de desempenho.
%		
%		\item Utiliza��o - � a porcentagem de tempo que um recurso est� ocupado. O percentual restante do tempo � considerado o tempo ocioso.

%	\end{itemize}

%\section{Normaliza��o dos Dados}
%\label{normal}
%	
%	A necessidade de normalizar os resultados se deu devido a an�lise dos tempos observados, que se mostraram, apesar de mesma grandeza (da ordem de segundos, ou no m�ximo minutos), serem diferentes quanto ao tempo de execu��o total, podendo um algoritmo executar em 50 segundos e outro em 8 minutos. Neste caso, houve a necessidade de executar o mesmo algoritmo (menor tempo de execu��o) diversas vezes para que houvesse concorr�ncia durante toda a execu��o do teste (algoritmo com execu��o mais longa). Com isso foi poss�vel extrair o percentual de aumento na execu��o concorrente do algoritmo de tempo mais longo e do algoritmo de tempo mais curto. Mas isto n�o foi suficiente para comparar individualmente cada algoritmo, sendo necess�rio criar um mecanismo que ``colocasse" todos os valores obtidos em uma mesma escala, e neste sentido optou-se pelo uso da normaliza��o de valores. Essa normaliza��o tem import�ncia para fornecer valores em uma escala pr�-definida que pode ser utilizada por escalonadores de tarefas para predizer o melhor recurso para executar determinado tipo de algoritmo, quando for necess�ria a exist�ncia de concorr�ncia.\\
%	
%	Para exemplificar melhor a normaliza��o de dados publicada por estes autores em \cite{licht:2013}, optou-se por apresentar recursos coletados de ordem de grandeza variadas em um n�vel onde os valores poder�o ser comparados mais facilmente, usou-se a F�rmula \ref{normalizacao}.\\
%\begin{equation}
%\label{normalizacao}
%	R = \frac{C - M2}{M1 - M2} (N1 - N2) + N2
%\end{equation}
%	Onde: \\
%	R : � o valor normalizado de cada entrada. \\
%	C : � o valor bruto coletado, sem tratamento. \\
%	M1 : � o maior valor encontrado em todos os valores lidos. \\
%	M2 : � o menor valor encontrado em todos os valores lidos. \\
%	N1 : � o maior valor da escala que se deseja trabalhar. \\
%	N2 : � o menor valor da escala que se deseja trabalhar. \\
%	
%	A maior vantagem do uso da F�rmula \ref{normalizacao} � poder elevar ou reduzir a pontua��o de tempos de acordo com a necessidade em cada medi��o e desta forma, j� que existe a compara��o entre todos os valores, a que tiver a maior pontua��o, � consequentemente a que tem a maior perda. Repara-se que o valor tratado de cada recurso poder� ser somado, j� que ap�s o uso da f�rmula, estar�o na mesma ordem de grandeza. Nesse exemplo, considera-se que os valores ter�o escala entre 0 e 9, sendo 0 para a menor perda de desempenho e 9 para a maior perda de desempenho.\\

%\begin{table}[h]
%\label{realnormal}
%\centering
%\caption{Tabela de Dados Reais Versus Dados Normalizados} % igual ao ambiente figura 
%\small
%\scriptsize
%\begin{tabular}{|c|c|c|c|c|c|c|c|c|c|c|c|c|c|}
%\hline
%\textbf{Algoritmo}  & \multicolumn{12}{c|}{\textbf{Tempo (segundos)}}                                 & \textbf{M�dia} \\ \hline
%Real B+Tree         & 14  & 14    & 14    & 14   & 14   & 14   & 14  & 14  & 14   & 14  & 14  & 14    & 14,05          \\ \hline
%V. B+Tree x V. SRAD & 221 & 218   & 220   & 223  & 219  & 219  & 221 & 225 & 223  & 217 & 221 & 218   & 220,6          \\ \hline
%R. B+Tree x V. LUD  & 21  & 21    & 22    & 20   & 20   & 21   & 21  & 22  & 21   & 21  & 21  & 21    & 21             \\ \hline
%V. B+Tree x R. SRAD & 211 & 214   & 212   & 216  & 212  & 213  & 215 & 211 & 213  & 214 & 211 & 214   & 213,1          \\ \hline
%\multicolumn{14}{|c|}{}                                                                                                \\ \hline
%\textbf{Algoritmo}  & \multicolumn{12}{c|}{\textbf{Tempo Normalizado}}                                & \textbf{M�dia} \\ \hline
%Real B+Tree         & 0   & 0     & 0     & 0    & 0    & 0    & 0   & 0   & 0    & 9   & 0   & 0     & 2,25           \\ \hline
%V. B+Tree x V. SRAD & 4,5 & 7,875 & 5,625 & 2,25 & 6,75 & 6,75 & 4,5 & 0   & 2,25 & 9   & 4,5 & 7,875 & 4,95           \\ \hline
%R. B+Tree x V. LUD  & 4,5 & 4,5   & 0     & 9    & 9    & 4,5  & 4,5 & 0   & 4,5  & 4,5 & 4,5 & 4,5   & 4,5            \\ \hline
%V. B+Tree x R. SRAD & 9   & 3,6   & 7,2   & 0    & 7,2  & 5,4  & 1,8 & 9   & 5,4  & 3,6 & 9   & 3,6   & 5,22           \\ \hline
%\end{tabular}
%\end{table}
%
%\begin{equation}
%\label{somatorio}
%	P = \frac{\sum _{k=1}^{r} x_{k}}{r} 
%\end{equation}
%	Onde: \\
%	P � a pontua��o obtida. \\
%	r � a quantidade de valores. \\
%	$x_{k}$ � cada recurso j� normalizado.\\
%	
%	Para verificar a aplica��o das f�rmulas foram tomadas 4 amostras de tempos de execu��o e a m�dia gerada por estes. � poss�vel comparar na Tabela 3.1 as 4 primeiras linhas os valores (em segundos) da execu��o de alguns algoritmos. Percebe-se, por exemplo, que a m�dia obtida por ``Real B+Tree" (14,05) tem uma ordem da valor bem menor que ``V. B+Tree x V. SRAD" (220,6), n�o podendo ser comparados diretamente, entretanto, ap�s normalizados, os valores aparecem como 2,25 e 4,95 respectivamente. Esta normaliza��o permite atribuir valores de forma que um escalonador poderia definir as melhores combina��es mediante uma escala de valores dentro de um intervalo pr�-definido, gerados pela F�rmula \ref{somatorio}.\\

%	Pode-se ainda, atrav�s da F�rmula \ref{somatorioComPesos} definir pesos para cada pontua��o afim de dar prioridade para determinadas combina��es, por exemplo, quando h� implementa��es com OpenMP e OpenCL (maior ou menor peso de acordo com o tipo de implementa��o).\\

%	As F�rmulas \ref{normalizacao}, \ref{somatorio} e \ref{somatorioComPesos} podem ainda ser utilizadas para outros tipos de dados, que referenciariam, capacidade de mem�ria, disco, cpu, ou qualquer outro tipo de valor de grandezas diferentes, por exemplo, para uma aplica��o que demanda maior quantidade de mem�ria, poder-se-ia atribuir um peso maior para servidores com mais mem�ria, esta entretanto � uma possibilidade a ser investigada em um novo projeto, n�o sendo contemplada por esta pesquisa.\\

%	\begin{equation}
%	\label{somatorioComPesos}
%		P = \frac{\sum _{k=1}^{r} x_{k} x p}{\sum_{k=1}^{r} p}
%	\end{equation}
%		Onde: \\
%		P � a pontua��o obtida por cada an�lise com os pesos de cada. \\
%		r � a quantidade de valores obtidos. \\
%		$x_{k}$ � cada valor j� normalizado.\\
%		p � o peso dado a cada item. \\


%	O uso da abordagem de normaliza��o definida aqui ser� melhor apresentada nos resultados da Se��o \ref{testesHeterogeneos}.

\section{Conclus�o do Cap�tulo}
%	Este cap�tulo apresentou as metodologias utilizadas, bem como os testes efetuados afim de propor o conceito de afinidade e a avalia��o dos crit�rios de coexist�ncia de tipos de aplica��es. No pr�ximo cap�tulo ser�o apresentados os testes e os resultados obtidos � partir destas metodologias.


%\chapter{An�lise dos resultados}
\label{analise:res}

% !TeX root = TCC.tex
\chapter{APRESENTA��O E DISCUSS�O DOS RESULTADOS}
\label{resultados}

Para as monografias emp�ricas, este cap�tulo refere-se a apresenta��o e discuss�o dos resultados da pesquisa. 
Deve-se retomar de modo sucinto os procedimentos metodol�gicos adotados (tipo de pesquisa, institui��o e sujeitos do estudo, instrumentos de coleta e an�lise de dados).

� importante situar o leitor do \textbf{campo} escolhido para realiza��o da pesquisa. Descreva a/as institui��o/�es de maneira objetiva: sua localiza��o, �rea de atua��o, dimens�es, estrutura f�sica e humana.
 
� preciso ainda \ul{\textbf{indicar e caracterizar os sujeitos do estudo}} (como se deu a sele��o dos participantes, os crit�rios de inclus�o e de exclus�o, sexo, idade, escolaridade, cargos/fun��es, etc).

A seguir, procure \ul{\textbf{detalhar as etapas de investiga��o}} de modo a que o leitor possa reconhecer todo o caminho percorrido: como foi a aproxima��o a institui��o/aos sujeitos, \ul{\textbf{quais os instrumentos de coleta de dados utilizados, o que voc� pretendia com cada instrumento e como analisou os dados coletados}}.

A seguir, em conson�ncia com os objetivos do estudo, devem ser apresentadas as categorias constru�das, as an�lises e interpreta��es feitas, todas devidamente ilustradas e fundamentadas. 

Cada autor tem um estilo redacional pr�prio, mas � fundamental que neste cap�tulo, as quest�es que de alguma forma nortearam a pesquisa possam ser respondidas. 

% !TEX root = dissertacao.tex
\chapter{Conclus�es}
	
\section{Considera��es Finais}

Os resultados obtidos no Cap�tulo \ref{Sec:resultados} se mostraram muito satisfat�rios. Neste mesmo cap�tulo foram reproduzidas duas imagens com os resultados para \gp (Figura \ref{Fig:resulado:PM}) e \dsw (Figura \ref{Fig:resulado:SI}), onde ressalto as seguintes observa��es:

\begin{itemize}
	
	\item Dentre os 5 primeiros itens priorizados para \gp, 3 est�o diretamente relacionados com o cliente (revis�o da qualidade do produto, aprova��o de solicita��es de mudan�as requisitadas pelo cliente e acompanhamento do projeto pelo cliente);
	
	\item Dentre os 5 primeiros itens priorizados para \dsw, 2 est�o diretamente relacionados com o cliente (entregas ao cliente e formaliza��o da manuten��o do \sw com o cliente);
	
	\item Os demais itens, das duas listas, est�o relacionados com processos que trar�o melhorias associadas indiretamente ao cliente.
	
\end{itemize}

\iftoggle{full}
{

	Esses n�meros indicam que, caso a organiza��o siga esta lista de prioriza��es e escolha os 5 primeiros itens para cada grupo, ela n�o somente ter� uma quantidade exequ�vel de a��es de melhorias de processos (10 processos em compara��o aos mais de 70 totais) como essas a��es iniciais ir�o abordar melhorias em �reas relevantes para a empresa ($50\%$ dos processos est�o ligados ao cliente, principal preocupa��o da organiza��o).

	Al�m disso, outros valores empresariais foram levados em considera��o nestas listas geradas pelo question�rio. Seus pesos influenciaram na prioriza��o e valores com pesos iguais ou pr�ximos ao atendimento ao cliente tamb�m posicionaram os processos associados no topo das listas.

	Levando em considera��o o objetivo desta disserta��o:
	\begin{quote}
		``Criar um m�todo de auto diagn�stico para implanta��o da \iso que priorize processos com maior ader�ncia aos valores individuais de cada empresa e que tragam benef�cios relevantes observ�veis nas etapas iniciais da implanta��o'' (Cap�tulo \ref{Sec:delim:trab})
	\end{quote} podemos afirmar que alcan�amos este objetivo.
	
}

A hip�tese levantada no Cap�tulo \ref{Sec:metodologia} afirmava que uma das melhores solu��es para o problema definido no Cap�tulo \ref{sec:def:prob} era a implanta��o de um question�rio de auto avalia��o. Analisando os resultados obtidos, podemos afirmar que o question�rio foi realmente uma ferramenta pr�tica e �til para a organiza��o montar sua estrat�gia de a��es de melhorias de processos.

\iftoggle{full}
{

	\section{Trabalhos Futuros}

	A fim de permitir uma maior flexibilidade na personaliza��o dos question�rios, o autor tem como meta futura transpor o question�rio de uma planilha eletr�nica para um \sw pr�prio. Conforme visto no Cap�tulo \ref{Sec:rev:pesos}, algumas dificuldades inerentes �s planilhas eletr�nicas tornam a tarefa de personaliza��o de alguns campos complexa e trabalhosa e um \sw aliado a um banco de dados trar�o benef�cios relevantes para este trabalho.

	Outro benef�cio da troca da planilha eletr�nica por um \sw espec�fico � a automa��o na procura pelo cen�rio ideal e refer�ncia � norma \iso para cada processo identificado na lista de melhorias. Pelas limita��es da planilha eletr�nica, o processo atual � manual e, consequentemente, exige mais tempo e trabalho do entrevistado. A partir de um \sw espec�fico integrado a um banco de dados, � poss�vel exibir o cen�rio ideal, nomenclatura original do processo na \iso e mais uma s�rie de informa��es, dicas e outras informa��es que auxiliem no planejamento e execu��o das melhorias. Ser� poss�vel realizar uma gest�o do conhecimento em cima dos processos, permitindo at� mesmo anexar documentos e refer�ncias externas para materiais de apoio. 

	Devido �s caracter�sticas do question�rio e das ideias de trabalhos futuros citadas anteriormente, uma solu��o desenhada para internet se mostra uma das melhores op��es. Al�m de maior disponibilidade, visto que o question�rio estaria dispon�vel em mais de um computador, tamb�m permite a utiliza��o de diferentes dispositivos, principalmente os m�veis como tablets e celulares.
	
}



%% !TEX root = dissertacao.tex

\appendix
\chapter{Diagn�stico da empresa}

\section{An�lise Organizacional e de Processos}
\label{Sec:analise:org}

\textbf{Diagn�stico:} observou-se que a empresa possui qualidades essenciais para o crescimento cont�nuo, como por exemplo, o comprometimento dos profissionais, a comunica��o, o bom clima organizacional, assim como a cultura de prezar pela excel�ncia e ser reconhecida atrav�s da sua confiabilidade e qualidade nos servi�os. Por�m, para que a empresa suporte o crescimento que tende a acontecer cada vez mais, devido a demanda pelos servi�os, tornam-se necess�rios alguns reajustes nos processos.

\subsection{Suporte T�cnico}
\label{Sec:suporte}

Este departamento � o �nico respons�vel pelo atendimento ao cliente atualmente. Os atendimentos podem ser realizados atrav�s do telefone, acesso remoto via internet, e-mail ou presencial. Como pode ser observado na Figura \ref{Fig:atend11}, somente 3 tipos foram registrados no m�s de novembro de 2014, mostrando que as comunicac�es via e-mail n�o s�o registradas.
%Assuntos sobre problemas e d�vidas relacionados aos \sws fornecidos s�o tratados diretamente com os t�cnicos do suporte. Para isso o departamento se utiliza de 2 \sws, cujas especifica��es est�o relacionadas na Tabela \ref{Tab:espec:sw:atend} e as janelas principais podem ser observadas nas Figuras \ref{Fig:jan:contatos:lanca} e \ref{Fig:jan:pendencias:lanca}.

%\begin{table}[h!]\footnotesize
%	\centering
%	\begin{tabular}
%		{
%			|p{1,5cm}|p{12cm}|
%		}
%		
%		\hline
%		\textbf{Nome}&
%		\textbf{Especifica��es}\\
%		\hline
%		
%		Contatos&
%		\begin{itemize}
%			\item Registra atendimentos;
%			\item Registra o cliente que est� sendo atendido;
%			\item Possui um espa�o virtualmente ilimitado para informar o assunto do contato;
%			\item Cronometra automaticamente o atendimento;
%			\item Possui uma data de previs�o de retorno;
%			\item N�o permite inclus�es de novas intera��es com o cliente durante o andamento do atendimento, que pode se estender por dias (neste caso novos atendimentos dever�o ser registrados, sem haver qualquer liga��o entre a primeira e as demais chamadas).
%		\end{itemize}\\
%		\hline
%		
%		Pend�ncias&
%		\begin{itemize}
%			\item Registra tarefas;
%			\item Registra o cliente que fez a solicita��o (opcional);
%			\item Associa o solicitante e o respons�vel por sua realiza��o (ambos colaboradores da empresa);
%			\item Permite definir prioridade, data limite para realiza��o e previs�o de atendimento;
%			\item Possui um controle de fechamento, onde o respons�vel indica a data em que aquela tarefa foi realizada;
%			\item Possui um controle de ``ok'', onde o solicitante indica que aquela tarefa j� foi liberada para o cliente;
%			\item Diferentemente do Contatos, permite acrescentar novas tarefas ao mesmo registro principal.
%		\end{itemize}\\
%		\hline
%		
%	\end{tabular}
%	\caption {Especifica��es dos \sws de atendimento}
%	\label{Tab:espec:sw:atend}
%\end{table}

%\begin{figure}[!h]
%	\centering
%	\includegraphics[scale=0.6]{figuras/jan_contatos_lancamento.jpg}
%	\caption{Janela de lan�amento dos atendimentos}
%	\label{Fig:jan:contatos:lanca}
%\end{figure}

%\begin{figure}[!h]
%	\centering
%	\includegraphics[scale=0.6]{figuras/jan_pendencias_lancamento.jpg}
%	\caption{Janela de lan�amento das pendencias}
%	\label{Fig:jan:pendencias:lanca}
%\end{figure}

\begin{figure}[!h]
	\centering
	\includegraphics[scale=0.5]{figuras/atendimentos_por_tipo.jpg}
	\caption{Atendimentos por tipo em novembro/2014}
	\label{Fig:atend11}
\end{figure}

Outros problemas foram diagnosticados e est�o relacionados na Tabela \ref{Tab:probl:atend}, sendo a falta de acompanhamento do andamento das solicita��es e retorno ao cliente as mais cr�ticas.

\begin{table}[h!]\footnotesize
	\centering
	\begin{tabular}
		{
			|p{14cm}|
		}
		
		\hline
		\textbf{Problema diagnosticado}\\
		\hline
		
		Falta de registro de atendimento (para qualquer tipo)\\
		\hline
		
		Posterga��o de registro de atendimento (para qualquer tipo), que pode levar ao esquecimento (falta de registro)\\
		\hline
		
		Dados insuficientes sobre o contato\\
		\hline
		
		Falta de acompanhamento do andamento das solicita��es (fechamento dos registros de atendimento)\\
		\hline
		
		Falta de retorno da situa��o das solicita��es ao cliente\\
		\hline
		
		Tarefas originadas dos atendimentos, para o pr�prio departamento de suporte ou para outros departamentos, s�o registradas em um \sw separado e n�o h� nenhuma rastreabilidade\\
		\hline
		
	\end{tabular}
	\caption {Problemas diagnosticados no departamento de suporte}
	\label{Tab:probl:atend}
\end{table}

%As seguintes a��es devem ser tomadas para contonar os problemas diagnosticados:
%
%\begin{itemize}
%	
%	\item Todos os chamados t�cnicos dever�o ser registrados no ato de sua execu��o (liga��o, acesso remoto ou recebimento do e-mail), com exce��o do atendimento externo que deve ser registrado no ato do retorno do t�cnico;
%	
%	\item O m�ximo de informa��es poss�veis precisa ser registrado no hist�rico do chamado;
%	
%	\item A fus�o entre o Contatos e o Pend�ncias � extremamente necess�ria, a fim de criar um registro de todas as a��es vinculadas naquele chamado, tornando-se obrigat�rio o registro de todas as a��es realizadas por cada t�cnico que atender o chamado;
%	
%	\item Qualquer nova informa��o de um chamado deve ser adicionada ao registro j� aberto, sem a necessidade de abrir um novo chamado e permitindo a rastreabilidade;
%	
%	\item Um departamento de Servi�o de Atendimento ao Cliente (SAC) dever� ser criado e pelo menos uma pessoa deve ser colocada como respons�vel pelas suas atribui��es;
%	
%	\item O SAC realizar� o fechamento do chamado junto ao cliente, consultando se realmente o problema foi solucionado e realizando pesquisas de satisfa��o e gerando indicadores de desempenho e qualidade (tempo m�dio de conclus�o dos chamados, problemas mais ocorridos, cliente mais ativo, t�cnico mais ativo, etc);
%	
%	\item Os clientes dever�o receber um Documento de Abertura e Acompanhamento de Chamados, juntamente com o manual  do \sw, para que saibam exatamente como proceder para abrir e acompanhar um chamado;
%	
%	\item Para uma melhor performance da equipe de suporte, dever� ser feito a gest�o do conhecimento, documentando a resolu��o de cada problema que surja no atendimento do suporte t�cnico;
%	
%	\item � necess�ria a elabora��o de um fluxograma do atendimento, com as informa��es chaves de requisitos para abertura de chamados;
%	
%	\item � necess�ria a elabora��o de um fluxograma do atendimento de servi�os diferenciados, tais como treinamentos, suportes avulsos, entrada de equipamento para concerto ou manuten��o, entre outros, lembrando de acrescentar no fluxo a emiss�o da Ordem de Servi�o;
%	
%	\item Dever� ser realizado o acompanhamento dos clientes que n�o abrem chamado com o suporte h� mais de 3 meses.
%	
%\end{itemize}

\subsection{Desenvolvimento}
\label{Sec:desenvolvimento}

Este departamento n�o tem contato direto com os clientes, pois todas as solicita��es passam pelo departamento de suporte t�cnico. Por�m, todas as solcita��es de mudan�a ou corre��o de problemas nos \sws s�o resolvidas por este departamento e alguns atendimentos s�o repassados para o setor de desenvolvimento para resolu��o conjunta quando os t�cnicos n�o possuem conhecimento ou capacidade para trat�-los por si mesmos.

Os problemas diagnosticados para este departamento est�o relacionados na Tabela \ref{Tab:probl:desenv}.

%As tarefas deste departamento s�o registradas no \sw Pend�ncias mas, como j� relatado anteriormente, n�o possuem relacionamento com os chamados abertos no \sw Contatos. 

\begin{table}[h!]\footnotesize
	\centering
	\begin{tabular}
		{
			|p{14cm}|
		}
		
		\hline
		\textbf{Problema diagnosticado}\\
		\hline
		
		Falta de posicionamento quanto ao andamento das solicita��es\\
		\hline
		
		Falta de previs�o de entrega das solu��es\\
		\hline
		
		Falta de rastreabilidade entre abertura de chamados e tarefas\\
		\hline
		
	\end{tabular}
	\caption {Problemas diagnosticados no departamento de desenvolvimento}
	\label{Tab:probl:desenv}
\end{table}

%Para solucionar os problemas diagnosticados para este setor, al�m das a��es ja citadas em \ref{Sec:suporte}, ser�o necess�rias as seguintes a��es:
%
%\begin{itemize}
%	
%	\item Criar processos para determinar a previs�o de lan�amento de vers�es de \sws;
%	
%	\item Criar processos para determinar em qual vers�o determinada solicita��o ser� inclu�da;
%	
%	\item Integrar aos \sws de atendimento as informa��es de lan�amento de vers�es e, consequentemente, a previs�o das solicitac�es.
%	
%\end{itemize}

\subsection{Financeiro}

O departamento financeiro lida com o cliente com uma frequ�ncia menor que os departamentos de suporte e desenvolvimento. Por�m, os assuntos relacionados a este departamento podem gerar transtornos e preju�zos quando feitos de forma incorreta. Al�m disso, este departamento tamb�m � respons�vel por bloquear o atendimento aos clientes inadimplentes, portanto representa um papel importante nos processos descritos em \ref{Sec:suporte} e \ref{Sec:desenvolvimento}.

O principal problema detectado neste departamento foi a falta de registro dos contatos realizados com o cliente.

%Nenhum atendimento ou tarefa relacionados a este departamento s�o registrados nos \sws de atendimento. Portanto, as a��es necess�rias para melhoria dos processos s�o:

%\begin{itemize}
%	
%	\item Registrar todo e qualquer contato com clientes nos \sws de atendimento;
%	
%	\item Integrar as informa��es financeiras com os \sws de atendimento para que a libera��o ou bloqueio seja feito automaticamente no ato da identifica��o do cliente.
%	
%\end{itemize}

\subsection{Marketing}

O departamento de marketing � o respons�vel, geralmente, pelo primeiro contato com o cliente. Durante as negocia��es de venda de \sw, � comum haver solicita��es de modifica��es ou acertos sobre configura��es, convers�es de dados e treinamentos. %Por�m, nenhuma dessas informa��es s�o registradas nos \sws de atendimento. O mesmo ocorre na venda de equipamentos e outros servi�os.

O principal problema detectado neste departamento foi a falta de registro dos contatos realizados com o cliente.

%Portanto, as a��es necess�rias para melhoria dos processos s�o:
%
%\begin{itemize}
%	
%	\item Registrar os primeiros contatos com todos os prospectos, mesmo que n�o se tornem clientes;
%	
%	\item Registrar toda e qualquer intera��o com os clientes, novos ou antigos;
%	
%	\item Incluir nos registros as informa��es sobre negocia��o, poss�veis modifica��es, convers�es e outras condi��es estabelecidas durante ou ap�s a venda.
%	 
%\end{itemize}

\chapter{An�lise SWOT}
\label{Sec:SWOT}

A fim de resumir e melhor entender a an�lise organizacional e de processos realizada em \ref{Sec:analise:org}, foi desenvolvida a Tabela \ref{Tab:SWOT} com o resultado da an�lise do cen�rio interno e externo da empresa. Esta an�lise de ambiente � conhecida como an�lise SWOT e realiza uma avalia��o das for�as, fraquezas, oportunidades e amea�as, dos termos em ingl�s \textit{strengths, weaknesses, opportunities e threats} \citep{kotler}.

\begin{table}[h!]\footnotesize
\centering
\begin{tabular}
{
	| >{\centering\arraybackslash}p{7cm}
	| >{\centering\arraybackslash}p{7cm}|
}
\hline

	For�as&
	Fraquezas\\
	\hline

	%For�as
	\begin{tabular}{p{6,8cm}}
	Comprometimento dos profissionais;\\
	Bom clima organizacional;\\
	Cultura de prezar pela excel�ncia;\\
	Ser reconhecida atrav�s da sua confiabilidade e qualidade nos servi�os.\\
	\end{tabular}&

	%Fraquezas
	\begin{tabular}{p{6,8cm}}
	Falta de informa��es no acompanhamento de solicita��es;\\
	Inefici�ncia na comunica��o com o cliente.\\
	\end{tabular}\\
	\hline

	Oportunidades&
	Amea�as\\
	\hline

	% Oportunidades
	\begin{tabular}{p{6,8cm}}
	Taxa de crescimento elevada nos �ltimos meses;\\
	Aumento no faturamento;\\
	Alta divulga��o da marca, dentro e fora da sua pr�pria cidade.\\
	\end{tabular}&

	% Amea�as
	\begin{tabular}{p{6,8cm}}
	Perder a confiabilidade dos clientes;\\
	Manchar a reputa��o;\\
	Sofrer queda nas vendas pela perda de indica��es de clientes e parceiros insatisfeitos;\\
	Sofrer perda de clientes j� estabelecidos por n�o prestar um bom atendimento.\\
	\end{tabular}\\
	\hline

\end{tabular}
\caption {An�lise SWOT dos processos}
\label{Tab:SWOT}
\end{table}

Ao se realizar uma avalia��o cr�tica dos resultados encontrados na an�lise SWOT, � poss�vel encontrar subs�dios que justificam o investimento de recursos em um projeto de melhoria de processos com base na Norma \iso para a empresa analisada.

A seguir ser�o analisados em mais detalhes os resultados da an�lise dos ambientes externo e interno.

\section{An�lise do ambiente externo}

A an�lise SWOT separa o ambiente externo em dois segmentos: oportunidades e amea�as. Ambos est�o fora do controle direto da organiza��o e s� podem ser influenciados a partir de a��es estrat�gicas para aumentar a probabilidade de ocorr�ncia dos eventos positivos (oportunidades) e tentar evitar a ocorr�ncia dos eventos negativos (amea�as).

As oportundidades resultantes da an�lise de cen�rio apontam um caminho de crescimento em quantidade de clientes e em valores de faturamento. Tal resultado pode ser considerado como cen�rio ideal e deve ser perseguido pela empresa.

As amea�as resultantes da an�lise de cen�rio mostram, em contrapartida ao cen�rio ideal apresentado pelas oportunidades, um cen�rio de perdas financeiras e de credibilidade. Tal resultado deve ser evitado e os fatores que podem levar a essas amea�as devem ser monitorados constantemente.

� poss�vel relacionar as oportunidades e amea�as, projetando uma deterioriza��o do cen�rio ideal tra�ado nas oportunidades, que � afetado pelo crescimento desordenado e sem processos formalizados que o possam suportar, fazendo com que diversos problemas aflorem e invertam o cen�rio de crescimento, levando a empresa ao cen�rio descrito nas amea�as. Esta an�lise foi citada no Cap�tulo \ref{Introducao} como um dos principais motivadores da cria��o deste projeto.

\section{An�lise do ambiente interno}

A an�lise SWOT separa o ambiente interno em dois segmentos: for�as e fraquezas. Diferentemente do ambiente externo, este dois segmentos est�o dentro do controle direto da organiza��o e s�o fontes das a��es estrat�gicas que ser�o tra�adas para se alcan�ar as oportunidades e evitar as amea�as.

As for�as resultantes da an�lise de cen�rio apontam um ambiente f�rtil para o crescimento, onde clima, cultura e comprometimento s�o apontados como pontos fortes da organiza��o.

As fraquezas resultantes da an�lise de cen�rio mostram como a falta de processos formalizados afetam o principal ativo da empresa: os clientes.

Um dos pontos positivos da an�lise SWOT � a constata��o de que o caminho para eliminar as fraquezas � potencializado pelas caracter�sticas internas apontadas nas for�as. Qualquer processo de melhoria empresarial depende crucialmente do engajamento da equipe de trabalho e, como mencionado anteriormente, o ambiente interno da organiza��o oferece um terreno f�rtil para esse fim.

\section{An�lise global}

A an�lise do sistema organzacional completo (ambientes externo e interno), em conjunto com as an�lises realizadas na Se��o \ref{Sec:analise:org}, mostram que as atitudes em rela��o �s fraquezas determinar�o quais dos dois poss�veis caminhos a empresa ir� trilhar: agarrar as oportunidades e alcan�ar o cen�rio ideal, ou ser abatida pelas amea�as e inverter o cen�rio de crescimento.

Como o objetivo da empresa � o crescimento e a elimina��o das amea�as, se tornou claro que era necess�rio um ajuste nos processos internos. Como j� citado no Cap�tulo \ref{Introducao}, a solu��o escolhida foi a implanta��o da Norma \iso.

O processo de implanta��o da Norma contempla duas etapas principais: um diagn�stico mais detalhado do cen�rio atual dos processos de desenvolvimento de \sw; e a cria��o de um projeto de implanta��o da Norma. Esta disserta��o contribuiu para a primeira etapa atrav�s do auto diagn�stico do cen�rio atual.

\chapter{Exemplo de question�rio para \gp}

\begin{table}[h!]\footnotesize
\begin{tabular}
	{
		| >{\centering\arraybackslash}p{1,8cm}
		| >{\centering\arraybackslash}p{1,8cm}
		| >{\centering\arraybackslash}p{1,8cm}
		| >{\centering\arraybackslash}p{1,8cm}
		| >{\centering\arraybackslash}p{1,8cm}
		| >{\centering\arraybackslash}p{1,8cm}
		| >{\centering\arraybackslash}p{1,8cm}|
	}	
	\hline
	PM.1.1.Q1 & O levantamento de requisitos � feito? & Sim e s�o registrados em formul�rios ou ferramentas padronizadas, para todos os projetos de SW & Sim, para todos os projetos de SW, mas os registros n�o s�o padronizados & Sim e s�o registrados em formul�rios ou ferramentas padronizadas, mas somente para projetos mais complexos ou valiosos & Sim, mas somente para projetos mais complexos ou valiosos, mas os registros n�o s�o padronizados & N�o \\ \hline
	&  &  &  &  & x &  \\ \hline
	PM.1.1.Q2 & Crit�rios de aceita��o dos requisitos s�o discriminados? & Sim, em conjunto com o cliente & Sim, somente pelo cliente & Sim, somente pelo GP/LE & Sim, somente pela equipe do projeto & N�o/N�o s�o levantados os requisitos \\ \hline
	&  &  &  &  &  & x \\ \hline
	PM.1.1.Q3 & A empresa possui um documento de abertura de projeto (Declara��o de Trabalho, Termo de Abertura do Projeto, etc.?) & Sim e utliza para todos os projetos de SW & Sim mas utiliza somente para os projetos de SW mais complexos ou mais valiosos & Sim mas utiliza somente para alguns projetos de SW ou somente algumas pessoas utilizam & N�o h� nada formalizado mas algum tipo de documenta��o � gerada no in�cio do projeto, variando em forma e conte�do a cada projeto e conforme a pessoa que o produz & N�o \\ \hline
	&  &  & x &  &  &  \\ \hline
	\end{tabular}
	\caption{Exemplo de question�rio para \gp}
\end{table}

\begin{table}[h!]\footnotesize
	\begin{tabular}
		{
			| >{\centering\arraybackslash}p{1,8cm}
			| >{\centering\arraybackslash}p{1,8cm}
			| >{\centering\arraybackslash}p{1,8cm}
			| >{\centering\arraybackslash}p{1,8cm}
			| >{\centering\arraybackslash}p{1,8cm}
			| >{\centering\arraybackslash}p{1,8cm}
			| >{\centering\arraybackslash}p{1,8cm}|
		}	
		\hline
	PM.1.1.Q4 & O documento de abertura do projeto de SW � revisado antes do in�cio do projeto? & Sim, pelo GP/LE, equipe e cliente & Sim, pelo GP/LE e equipe & Sim, somente pelo GP/LE ou pela equipe e cliente & Somente para projetos mais complexos ou mais valiosos & N�o \\ \hline
	&  &  &  &  &  & x \\ \hline
	PM.1.1.Q5 & O projeto de SW � decomposto em itens que ser�o entregues (ENTREG�VEIS) ao cliente? Exemplos: backlog, pacotes de trabalho, itens de trabalho (work items), etc. & Sim, o projeto � decomposto em componentes de SW (sistemas, bibliotecas, sites, etc.), documenta��es (manual, help, instru��es, etc.), itens do projeto (planos, relat�rios, etc.) entre outros & Sim, o projeto � decomposto em componentes de SW (sistemas, bibliotecas, sites, etc.) e documenta��es (manual, help, instru��es, etc.) & Sim, o projeto � decomposto em componentes de SW (sistemas, bibliotecas, sites, etc.) & Alguns entreg�veis s�o especificados somente em projetos mais complexos ou valiosos ou somente algumas pessoas fazem isso & N�o \\ \hline
	&  &  &  &  &  & x \\ \hline
	PM.1.2.Q1 & S�o definidas instru��es de entrega para os entreg�veis? & Sim, para todos os entreg�veis, definidas em conjunto com o cliente & Sim, para alguns dos entreg�veis, definidas em conjunto com o cliente & Sim, para todos os entreg�veis, definidas somente pelo GP/LE ou equipe & Sim, para alguns dos entreg�veis, definidas somente pelo GP/LE ou equipe & N�o \\ \hline
	&  &  &  &  &  & x \\ \hline
	PM.1.3.Q1 & � gerada uma lista de atividades para desenvolver o SW e prodzir outros entreg�veis do projeto? Exemplos: tarefas (tasks), pend�ncias, etc. & Sim, cada entreg�vel possui uma ou mais atividades associadas e elas s�o levantadas na fase de planejamento & Sim, cada entreg�vel possui uma ou mais atividades associadas e elas s�o levantadas na fase de execu��o & Sim mas somente para alguns projetos de SW mais complexos ou mais valiosos, durante a fase de planejamento & Sim mas somente para alguns projetos de SW mais complexos ou mais valiosos, durante a fase de execu��o & N�o \\ \hline
	&  &  & x &  &  &  \\ \hline
	\end{tabular}
\end{table}

\begin{table}[h!]\footnotesize
	\begin{tabular}
		{
			| >{\centering\arraybackslash}p{1,8cm}
			| >{\centering\arraybackslash}p{1,8cm}
			| >{\centering\arraybackslash}p{1,8cm}
			| >{\centering\arraybackslash}p{1,8cm}
			| >{\centering\arraybackslash}p{1,8cm}
			| >{\centering\arraybackslash}p{1,8cm}
			| >{\centering\arraybackslash}p{1,8cm}|
		}	
		\hline
	PM.1.3.Q2 & O processo de desenvolvimento de SW prev� atividades de verifica��o, valida��o e revis�o para garantir a qualidade do produto & Sim, unindo a equipe, GP/LE e cliente & Sim, somente entre GP/LE e cliente & Sim, somente para a equipe & Somente em casos de impedimentos/urg�ncia & N�o \\ \hline
	&  &  &  &  &  & x \\ \hline
	PM.1.3.Q3 & Atividades necess�rias para executar as instru��es de entrega s�o identificadas? & Sim, durante o planejamento & Sim, durante o desenvolvimento & Sim, durante as valida��es & Sim, somente quando necess�rio, pr�ximo � entrega & N�o \\ \hline
	&  &  &  &  &  & x \\ \hline
	PM.1.4.Q1 & A dura��o ou o esfor�o para realizar cada tarefa � estimado & Sim, durante o planejamento inicial, em conjunto com toda a equipe & Sim, durante o planejamento de cada etapa (sprint, ciclo, itera��o, etc.), em conjunto com toda a equipe & Sim, durante o planejamento inicial, individualmente ou pelo GP/LE & Sim, durante o planejamento de cada etapa (sprint, ciclo, etc.), individualmente ou pelo GP/LE & N�o \\ \hline
	&  &  &  & x &  &  \\ \hline
	PM.1.5.Q1 & As pessoas necess�rias para executar o projeto s�o identificadas? & Sim, durante o planejamento do projeto & Sim, durante o planejamento de cada etapa (sprint, ciclo, itera��o, etc.) & N�o, a designa��o � sob demanda, no in�cio de cada atividade & N�o, a equipe � muito pequena e todos trabalham no projeto ao mesmo tempo, executando as atividades sob demanda & N�o \\ \hline
	&  &  &  &  &  & x \\ \hline
	\end{tabular}
\end{table}

\begin{table}[h!]\footnotesize
	\begin{tabular}
		{
			| >{\centering\arraybackslash}p{1,8cm}
			| >{\centering\arraybackslash}p{1,8cm}
			| >{\centering\arraybackslash}p{1,8cm}
			| >{\centering\arraybackslash}p{1,8cm}
			| >{\centering\arraybackslash}p{1,8cm}
			| >{\centering\arraybackslash}p{1,8cm}
			| >{\centering\arraybackslash}p{1,8cm}|
		}	
		\hline
	PM.1.5.Q2 & Os recursos necess�rios para executar o projeto s�o identificadas? Exemplo: equipamentos, salas para treinamentos, servidores, ferramentas, etc. & Sim, durante o planejamento do projeto & Sim, durante o planejamento de cada etapa (sprint, ciclo, itera��o, etc.) & N�o, os recursos s�o obtidos sob demanda, no in�cio de cada atividade & N�o, a estrutura � muito pequena e todos trabalham no projeto ao mesmo tempo, obtendo os recursos sob demanda & N�o \\ \hline
	&  &  &  &  &  & x \\ \hline
	PM.1.5.Q3 & Os padr�es necess�rios para executar o projeto s�o identificadas? Exemplo: procedimentos padr�es de desenvolvimento, normas internas, manuais de melhores pr�ticas, etc. & Sim, durante o planejamento do projeto & Sim, durante o planejamento de cada etapa (sprint, ciclo, itera��o, etc.) & N�o, mas todos da equipe receberam instru��es pr�vias, escritas ou orais, de como executar o projeto & N�o, mas todos da equipe receberam algum tipo de treinamento ao entrar na empresa de como executar projetos de SW & N�o \\ \hline
	&  &  &  &  &  & x \\ \hline
	PM.1.5.Q4 & Quando necess�rio, os treinamentos da equipe essenciais para a execu��o do projeto s�o identificados? & Sim, durante o planejamento do projeto & Sim, durante o planejamento de cada etapa (sprint, ciclo, itera��o, etc.) & N�o, treinamentos s�o executados sob demanda mas s�o considerados parte do projeto & N�o, treinamentos n�o s�o considerados parte do projeto e s�o executados sob demanda & Treinamentos n�o s�o fornecidos para a equipe \\ \hline
	&  &  &  &  & x &  \\ \hline
	\end{tabular}
\end{table}

\begin{table}[h!]\footnotesize
	\begin{tabular}
		{
			| >{\centering\arraybackslash}p{1,8cm}
			| >{\centering\arraybackslash}p{1,8cm}
			| >{\centering\arraybackslash}p{1,8cm}
			| >{\centering\arraybackslash}p{1,8cm}
			| >{\centering\arraybackslash}p{1,8cm}
			| >{\centering\arraybackslash}p{1,8cm}
			| >{\centering\arraybackslash}p{1,8cm}|
		}	
		\hline
	PM.1.5.Q5 & As datas em que as pessoas ser�o requisitadas para o projeto s�o inseridas no cronograma? & Sim, durante o planejamento do projeto & Sim, durante o planejamento de cada etapa (sprint, ciclo, itera��o, etc.) & N�o, a designa��o � sob demanda, no in�cio de cada atividade & N�o, a equipe � muito pequena e todos trabalham no projeto ao mesmo tempo, executando as atividades sob demanda & N�o \\ \hline
	&  &  &  &  &  & x \\ \hline
	PM.1.5.Q6 & As datas dos treinamentos necess�rios para o projeto s�o inseridas no cronograma? & Sim, durante o planejamento do projeto & Sim, durante o planejamento de cada etapa (sprint, ciclo, itera��o, etc.) & N�o, treinamentos s�o executados sob demanda mas s�o considerados parte do projeto & N�o, treinamentos n�o s�o considerados parte do projeto e s�o executados sob demanda & Treinamentos n�o s�o fornecidos para a equipe \\ \hline
	&  &  &  &  &  & x \\ \hline
	PM.1.6.Q1 & A montagem da equipe do projeto � planejada? & Sim, na fase de planejamento do projeto, de acordo com os recursos dispon�veis & Sim, na fase de planejamento do projeto, mas sem levar em considera��o os recursos dispon�veis & Sim, na fase de planejamento de cada etapa (sprint, ciclo, itera��o, etc.) & N�o, ela � montada sob demanda, no in�cio de cada atividade & N�o, a equipe � muito pequena e todos trabalham no projeto ao mesmo tempo, executando as atividades sob demanda \\ \hline
	&  &  &  &  &  & x \\ \hline
	PM.1.6.Q2 & As responsabilidades e pap�is de cada membro da equipe s�o planejadas? & Sim, na fase de planejamento do projeto, de acordo com os recursos dispon�veis & Sim, na fase de planejamento do projeto, mas sem levar em considera��o os recursos dispon�veis & Sim, na fase de planejamento de cada etapa (sprint, ciclo, itera��o, etc.) & N�o, responsabilidades e pap�is s�o atribu�dos sob demanda & N�o h� atribui��o clara de responsabilidades e pap�is \\ \hline
	&  &  &  & x &  &  \\ \hline
	\end{tabular}
\end{table}

\begin{table}[h!]\footnotesize
	\begin{tabular}
		{
			| >{\centering\arraybackslash}p{1,8cm}
			| >{\centering\arraybackslash}p{1,8cm}
			| >{\centering\arraybackslash}p{1,8cm}
			| >{\centering\arraybackslash}p{1,8cm}
			| >{\centering\arraybackslash}p{1,8cm}
			| >{\centering\arraybackslash}p{1,8cm}
			| >{\centering\arraybackslash}p{1,8cm}|
		}	
		\hline
	PM.1.7.Q1 & As datas de in�cio e t�rmino das atividades s�o estimadas? & Sim, na fase de planejamento do projeto, utilizando as estimativas de tempo/esfor�o & Sim, na fase de planejamento, sem utilizar estimativas anteriores de tempo e esfor�o & Sim, na fase de planejamento de cada etapa (sprint, ciclo, itera��o, etc.), utilizando as estimativas de tempo/esfor�o & Sim, na fase de planejamento de cada etapa (sprint, ciclo, itera��o, etc.), sem utilizar estimativas anteriores de tempo/esfor�o & N�o \\ \hline
	&  &  &  & x &  &  \\ \hline
	PM.1.7.Q2 & A disponibilidade dos recursos humanos e materiais � levada em considera��o quando as estimativas de cronograma s�o efetuadas? Exemplo: feriados, f�rias, parada para manuten��o & Sim, durante a fase de planejamento do projeto todos os eventos que afetam os recursos, humanos ou materiais, s�o inseridos no cronograma & Sim, durante a fase de planejamento do projeto, mas somente os eventos que afetam os recursos humanos s�o considerados & Sim, na fase de planejamento de cada etapa (sprint, ciclo, itera��o, etc.) todos os eventos que afetam os recursos, humanos ou materiais, s�o inseridos no cronograma & Sim, na fase de planejamento de cada etapa (sprint, ciclo, itera��o, etc.), mas somente os eventos que afetam os recursos humanos s�o considerados & N�o \\ \hline
	&  &  &  &  &  & x \\ \hline
	PM.1.7.Q3 & A rela��o entre as atividades � considerada na hora de montar o cronograma? & Sim, a sequ�ncia e a depend�ncia entre as atividades s�o analisadas e consideradas na fase de planejamento do projeto & Sim, a sequ�ncia e a depend�ncia entre as atividades s�o analisadas e consideradas na fase de planejamento de cada etapa (sprint, ciclo, itera��o, etc.) & Sim, mas somente a sequ�ncia entre as atividades � analisada e considerada na fase de planejamento do projeto & Sim, mas somente a sequ�ncia entre as atividades � analisada e considerada na fase de planejamento de cada etapa (sprint, ciclo, itera��o, etc.) & N�o \\ \hline
	&  &  &  &  &  & x \\ \hline
	\end{tabular}
\end{table}

\begin{table}[h!]\footnotesize
	\begin{tabular}
		{
			| >{\centering\arraybackslash}p{1,8cm}
			| >{\centering\arraybackslash}p{1,8cm}
			| >{\centering\arraybackslash}p{1,8cm}
			| >{\centering\arraybackslash}p{1,8cm}
			| >{\centering\arraybackslash}p{1,8cm}
			| >{\centering\arraybackslash}p{1,8cm}
			| >{\centering\arraybackslash}p{1,8cm}|
		}	
		\hline
	PM.1.8.Q1 & O esfor�o total requerido para terminar o projeto � calculado? & Sim, na fase de planejamento do projeto, utilizando as atividades e recursos levantados & N�o, somente para cada etapa (sprint, ciclo, itera��o, etc.) utilizando atividades e recursos & Sim, utilizando estimativas levantadas pela equipe & Sim, utilizando estimativas levantadas pelo GP/LE & N�o \\ \hline
	&  &  &  & x &  &  \\ \hline
	PM.1.8.Q2 & O custo total requerido para terminar o projeto � calculado? & Sim, na fase de planejamento do projeto, utilizando as atividades e recursos levantados & N�o, somente para cada etapa (sprint, ciclo, itera��o, etc.) utilizando atividades e recursos & Sim, utilizando estimativas levantadas pela equipe & Sim, utilizando estimativas levantadas pelo GP/LE & N�o \\ \hline
	&  &  &  &  &  & x \\ \hline
	PM.1.9.Q1 & Riscos s�o identificados e documentados? & Sim, durante o planejamento do projeto & Sim, durante o planejamento de cada etapa (sprint, ciclo, itera��o, etc.) & Sim, durante a execu��o do projeto & Sim, durante a execu��o do projeto, mas n�o s�o documentados & N�o, riscos n�o s�o considerados no projeto \\ \hline
	&  &  &  &  &  & x \\ \hline
	PM.1.10.Q1 & Existe uma estrat�gia de controle de versionamento? & Sim, para todos os artefatos do projeto e � documentada na fase de planejamento do projeto & Sim, somente para os artefatos de SW do projeto e � documentada na fase de planejamento do projeto & Sim, para todos os artefatos do projeto mas n�o faz parte da documenta��o do projeto & Sim, somente para os artefatos de SW do projeto mas n�o faz parte da documenta��o do projeto & N�o \\ \hline
	&  &  &  &  & x &  \\ \hline
	\end{tabular}
\end{table}

\begin{table}[h!]\footnotesize
	\begin{tabular}
		{
			| >{\centering\arraybackslash}p{1,8cm}
			| >{\centering\arraybackslash}p{1,8cm}
			| >{\centering\arraybackslash}p{1,8cm}
			| >{\centering\arraybackslash}p{1,8cm}
			| >{\centering\arraybackslash}p{1,8cm}
			| >{\centering\arraybackslash}p{1,8cm}
			| >{\centering\arraybackslash}p{1,8cm}|
		}	
		\hline
	PM.1.11.Q1 & � gerado algum plano de projeto? & Sim, integrando todos os elementos do projeto (termo de abertura, instru��es de entrega, atividades, cronograma, composi��o da equipe, custos, riscos e estrat�gia de controle de vers�o) & Sim, integrando somente os principais elementos do projeto & Sim, mas com informa��es n�o integradas & N�o, mas todos ou os principais elementos do projeto s�o disponibilizados para a equipe & N�o \\ \hline
	&  &  &  &  &  & x \\ \hline
	PM.1.12.Q1 & A descri��o do produto � criada? & Sim, e ap�s ser documentada � integrada ao plano do projeto & Sim, mas apesar de documentada, n�o � integrada ao plano do projeto & Sim, mas sem documenta��o formal, e disponibilizada para toda a equipe & Sim, mas sem documenta��o formal, e disponibilizada somente para o GP/LE & N�o \\ \hline
	&  &  &  &  &  & x \\ \hline
	PM.1.12.Q2 & O escopo do projeto � criado? & Sim, e ap�s ser documentado � integrado ao plano do projeto & Sim, mas apesar de documentado, n�o � integrado ao plano do projeto & Sim, mas sem documenta��o formal, e disponibilizado para toda a equipe & Sim, mas sem documenta��o formal, e disponibilizado somente para o GP/LE & N�o \\ \hline
	&  &  &  &  & x &  \\ \hline
	PM.1.12.Q3 & Os objetivos do projeto s�o levantados? & Sim, e ap�s serem documentados s�o integrados ao plano do projeto & Sim, mas apesar de documentados, n�o s�o integrados ao plano do projeto & Sim, mas sem documenta��o formal, e disponibilizados para toda a equipe & Sim, mas sem documenta��o formal, e disponibilizados somente para o GP/LE & N�o \\ \hline
	&  &  &  &  &  & x \\ \hline
	PM.1.12.Q4 & Os entreg�veis do projeto s�o identificados? & Sim, e ap�s serem documentados s�o integrados ao plano do projeto & Sim, mas apesar de documentados, n�o s�o integrados ao plano do projeto & Sim, mas sem documenta��o formal, e disponibilizados para toda a equipe & Sim, mas sem documenta��o formal, e disponibilizados somente para o GP/LE & N�o \\ \hline
	&  &  &  &  &  & x \\ \hline
	\end{tabular}
\end{table}

\begin{table}[h!]\footnotesize
	\begin{tabular}
		{
			| >{\centering\arraybackslash}p{1,8cm}
			| >{\centering\arraybackslash}p{1,8cm}
			| >{\centering\arraybackslash}p{1,8cm}
			| >{\centering\arraybackslash}p{1,8cm}
			| >{\centering\arraybackslash}p{1,8cm}
			| >{\centering\arraybackslash}p{1,8cm}
			| >{\centering\arraybackslash}p{1,8cm}|
		}	
		\hline
	PM.1.13.Q1 & � realizada uma an�lise de viabilidade do projeto? & Sim, todos os elementos do projeto s�o verificados quanto � viabilidade e consist�ncia, sendo os resultados documentados & Sim, todos os elementos do projeto s�o verificados quanto � viabilidade e consist�ncia, mas os resultados n�o s�o documentados & Sim, mas somente alguns elementos do projeto s�o verificados quanto � viabilidade e consist�ncia, sendo os resultados documentados & Sim, mas somente alguns elementos do projeto s�o verificados quanto � viabilidade e consist�ncia, mas os resultados n�o s�o documentados & N�o \\ \hline
	&  &  &  &  &  & x \\ \hline
	PM.1.13.Q2 & Como s�o tratados os problemas e inconsist�ncias encontrados na an�lise de viabilidade do projeto? & S�o documentados e corrigidos, gerando altera��es nos demais documentos e planos do projeto afetados & S�o documentados e corrigidos, mas os documentos e planos do projeto afetados permanecem inalterados & S�o corrigidos, mas n�o s�o documentados & N�o s�o corrigidos & N�o � feita a an�lise de viabilidade do projeto \\ \hline
	&  &  &  &  &  & x \\ \hline
	PM.1.13.Q3 & O plano de projeto � validado? & Sim, existe um procedimento formal de valida��o do plano de projeto pelo GP/LE antes do in�cio do projeto & Sim, formalmente pelo GP/LE, mas n�o existe um procedimento padr�o e a valida��o pode ocorrer at� mesmo com o projeto em andamento & Sim, informalmente pelo GP/LE, antes do in�cio do projeto & Sim, informalmente pelo GP/LE, a qualquer momento durante o projeto & N�o \\ \hline
	&  &  &  &  &  & x \\ \hline
	\end{tabular}
\end{table}

\begin{table}[h!]\footnotesize
	\begin{tabular}
		{
			| >{\centering\arraybackslash}p{1,8cm}
			| >{\centering\arraybackslash}p{1,8cm}
			| >{\centering\arraybackslash}p{1,8cm}
			| >{\centering\arraybackslash}p{1,8cm}
			| >{\centering\arraybackslash}p{1,8cm}
			| >{\centering\arraybackslash}p{1,8cm}
			| >{\centering\arraybackslash}p{1,8cm}|
		}	
		\hline
	PM.1.15.Q1 & O plano de projeto � revisado? & Sim, existe um procedimento formal de revis�o e aprova��o do plano de projeto pelo cliente antes do in�cio do projeto, confrontando com os elementos do documento de abertura do projeto & Sim, existe um procedimento formal de revis�o e aprova��o do plano de projeto somente pelo GP/LE antes do in�cio do projeto, confrontando com os elementos do documento de abertura do projeto & Sim, informalmente pelo cliente, antes do in�cio do projeto & Sim, informalmente pelo GP/LE, a qualquer momento durante o projeto & N�o \\ \hline
	&  &  &  &  &  & x \\ \hline
	PM.1.16.Q1 & Existe um reposit�rio do projeto? & Sim, definido no planejamento do projeto, documentado e mantido pela estrat�gia de controle de vers�o & Sim, definido no planejamento do projeto mas sem controle de vers�o & Sim, um reposit�rio padr�o para todos os projetos da empresa e mantido pela estrat�gia de controle de vers�o & Sim, um reposit�rio padr�o para todos os projetos da empresa mas sem controle de vers�o & N�o \\ \hline
	&  &  &  & x &  &  \\ \hline
	\end{tabular}
\end{table}

\chapter{Exemplo de question�rio para \dsw}

\begin{table}[h!]\footnotesize
	\begin{tabular}
		{
			| >{\centering\arraybackslash}p{1,8cm}
			| >{\centering\arraybackslash}p{1,8cm}
			| >{\centering\arraybackslash}p{1,8cm}
			| >{\centering\arraybackslash}p{1,8cm}
			| >{\centering\arraybackslash}p{1,8cm}
			| >{\centering\arraybackslash}p{1,8cm}
			| >{\centering\arraybackslash}p{1,8cm}|
		}	
		\hline
		SI.2.1.Q1 & Como as tarefas s�o distribu�das pela equipe? & De acordo com os pap�is dos membros da equipe, baseado no plano de projeto & De acordo com o plano de projeto, mas n�o existem pap�is definidos & De acordo com os pap�is dos membros da equipe, mas n�o existe um plano de projeto & O GP/LE distribui as tarefas conforme a demanda e os pap�is definidos & O GP/LE distribui as tarefas conforme a demanda mas os pap�is n�o foram definidos \\ \hline
		&  &  &  &  & x &  \\ \hline
		SI.2.2.Q1 & Como os requisitos s�o levantados? & Identificando e consultando clientes, usu�rios, sistemas anteriores, documentos, etc. & Identificando e consultando somente clientes, usu�rios e documentos & Identificando e consultando somente clientes e usu�rios & Identificando e consultando somente sistemas anteriores e/ou documentos & N�o s�o levantados os requisitos \\ \hline
		&  & x &  &  &  &  \\ \hline
		SI.2.2.Q2 & Como s�o analisados os requisitos? & S�o analisados para se determinar o escopo e viabilidade do projeto & S�o analisados para se determinar somente o escopo do projeto & S�o analisados para se determinar somente a viabilidade do projeto & S�o levantados mas n�o s�o analiados & N�o s�o levantados os requisitos \\ \hline
		&  &  &  & x &  &  \\ \hline
		SI.2.2.Q3 & Como s�o documentados os requisitos? & Em um documento de especifica��o de requisitos, integrado ao plano de projeto & Em um documento de especifica��o de requisitos, mas que n�o � integrado ao plano de projeto & Espalhados em diversos documentos (e-mails, relat�rios, entrevistas, etc.) disponibilizados � equipe & Espalhados em diversos documentos (e-mails, relat�rios, entrevistas, etc.) dispon�veis somente ao GP/LE & N�o s�o levantados os requisitos \\ \hline
		&  &  &  & x &  &  \\ \hline
	\end{tabular}
	\caption{Exemplo de question�rio para \dsw}
\end{table}

\begin{table}[h!]\footnotesize
	\begin{tabular}
		{
			| >{\centering\arraybackslash}p{1,8cm}
			| >{\centering\arraybackslash}p{1,8cm}
			| >{\centering\arraybackslash}p{1,8cm}
			| >{\centering\arraybackslash}p{1,8cm}
			| >{\centering\arraybackslash}p{1,8cm}
			| >{\centering\arraybackslash}p{1,8cm}
			| >{\centering\arraybackslash}p{1,8cm}|
		}	
		\hline
		SI.2.3.Q1 & Como s�o verificados os requisitos? & Os requisitos s�o verificados pelo analista em termos de corretude e testabilidade e quanto a sua consist�ncia com a descri��o do produto. S�o verificados para que estejam completos e n�o sejam amb�guos ou contradit�rios & Os requisitos s�o verificados pelo analista em termos de corretude e testabilidade, mas n�o quanto a sua consist�ncia com a descri��o do produto. S�o verificados para que estejam completos e n�o sejam amb�guos ou contradit�rios & Os requisitos s�o verificados pelo analista esporadicamente ou sem um processo formal & Os requisitos n�o s�o validados & N�o s�o levantados os requisitos \\ \hline
		&  &  &  &  & x &  \\ \hline
		SI.2.3.Q2 & Como s�o registrados os resultados da an�lise e verifica��o dos requisitos? & Resultados positivos s�o registrados em documento de resultado de valida��o e as corre��es necess�rias s�o feitas at� a verifica��o dos requisitos. Caso mudan�as significativas sejam necess�rias, uma requisi��o de mudan�as � criada. & Corre��es necess�rias s�o feitas at� a verifica��o dos requisitos. Caso mudan�as significativas sejam necess�rias, uma requisi��o de mudan�as � criada. & Corre��es necess�rias s�o feitas at� a verifica��o dos requisitos & Os requisitos n�o s�o verificados & N�o s�o levantados os requisitos \\ \hline
		&  &  &  &  & x &  \\ \hline
	\end{tabular}
\end{table}

\begin{table}[h!]\footnotesize
	\begin{tabular}
		{
			| >{\centering\arraybackslash}p{1,8cm}
			| >{\centering\arraybackslash}p{1,8cm}
			| >{\centering\arraybackslash}p{1,8cm}
			| >{\centering\arraybackslash}p{1,8cm}
			| >{\centering\arraybackslash}p{1,8cm}
			| >{\centering\arraybackslash}p{1,8cm}
			| >{\centering\arraybackslash}p{1,8cm}|
		}	
		\hline
		SI.2.4.Q1 & Como s�o validados os requisitos? & Os requisitos s�o validados pelo cliente levando-se em considera��o acordos firmados e suas necessidades e expectativas, incluindo usabilidade da interface com o usu�rio & Os requisitos s�o validados pelo cliente levando-se em considera��o somente acordos firmados & Os requisitos s�o validados pelo GP/LE & Os requisitos n�o s�o validados & N�o s�o levantados os requisitos \\ \hline
		&  &  &  &  & x &  \\ \hline
		SI.2.4.Q2 & Como s�o registrados os resultados da an�lise e valida��o dos requisitos? & Resultados positivos s�o registrados em um documento de resultado de valida��o e as corre��es necess�rias s�o feitas at� a valida��o dos requisitos & Corre��es necess�rias s�o feitas at� a valida��o dos requisitos & Corre��es necess�rias s�o feitas e a valida��o dos requisitos � postergarda at� mesmo para durante o desenvolvimento & Os requisitos n�o s�o validados & N�o s�o levantados os requisitos \\ \hline
		&  &  &  &  & x &  \\ \hline
		SI.2.7.Q1 & Onde e como � armazenada a especifica��o de requisitos? & Como um item de configura��o de software em uma baseline no reposit�rio do projeto & Como um item de configura��o de software em uma baseline, mas n�o existe um reposit�rio do projeto & Em uma baseline no reposit�rio do projeto & N�o existe um processo ou local espec�fico para armazenar os requisitos & N�o s�o levantados os requisitos \\ \hline
		&  &  &  &  & x &  \\ \hline
	\end{tabular}
\end{table}


%\bibliographystyle{brazil}
\bibliography{pesquisa}
\bibliographystyle{apalike-pt}
\addcontentsline{toc}{chapter}{\MakeUppercase{Bibliografia}}

\singlespacing

\end{document}