% !TEX root = dissertacao.tex
\chapter{Conclus�es}
	
Nesta se��o devem ser apresentadas as conclus�es a que se pode chegar com o estudo realizado, tendo em vista \textbf{responder � problematiza��o que motivou a pesquisa}. Admite posicionamentos e reflex�es mais pessoais.

Pode-se iniciar o texto resgatando o tema e os objetivos da pesquisa. Em seguida apresentar uma s�ntese dos principais resultados, evidenciando a import�ncia de cada um deles. Neste momento o pesquisador pode se posicionar questionando ou denunciando o valor negativo ou positivo dos resultados, fazer indica��es de novas problem�ticas a serem investigadas e/ou recomenda��es. 

Deve-se evitar recorrer a cita��es, � se��o de car�ter eminentemente autoral.
� de bom tom, assumir os limites que o estudo encerra, o car�ter provis�rio dos resultados e finalizar o texto com par�grafo propositivo frente ao problema que deu origem a pesquisa.
