% !TeX root = TCC.tex
\chapter{APRESENTA��O E DISCUSS�O DOS RESULTADOS}
\label{resultados}

Para as monografias emp�ricas, este cap�tulo refere-se a apresenta��o e discuss�o dos resultados da pesquisa. 
Deve-se retomar de modo sucinto os procedimentos metodol�gicos adotados (tipo de pesquisa, institui��o e sujeitos do estudo, instrumentos de coleta e an�lise de dados).

� importante situar o leitor do \textbf{campo} escolhido para realiza��o da pesquisa. Descreva a/as institui��o/�es de maneira objetiva: sua localiza��o, �rea de atua��o, dimens�es, estrutura f�sica e humana.
 
� preciso ainda \ul{\textbf{indicar e caracterizar os sujeitos do estudo}} (como se deu a sele��o dos participantes, os crit�rios de inclus�o e de exclus�o, sexo, idade, escolaridade, cargos/fun��es, etc).

A seguir, procure \ul{\textbf{detalhar as etapas de investiga��o}} de modo a que o leitor possa reconhecer todo o caminho percorrido: como foi a aproxima��o a institui��o/aos sujeitos, \ul{\textbf{quais os instrumentos de coleta de dados utilizados, o que voc� pretendia com cada instrumento e como analisou os dados coletados}}.

A seguir, em conson�ncia com os objetivos do estudo, devem ser apresentadas as categorias constru�das, as an�lises e interpreta��es feitas, todas devidamente ilustradas e fundamentadas. 

Cada autor tem um estilo redacional pr�prio, mas � fundamental que neste cap�tulo, as quest�es que de alguma forma nortearam a pesquisa possam ser respondidas. 
