%:
% !TEX TS-program = pdflatex
% !TEX encoding = UTF-8 Unicode

% This is a simple template for a LaTeX document using the "article" class.
% See "book", "report", "letter" for other types of document.

%\documentclass{scrartcl}
\documentclass[11pt]{article} % use larger type; default would be 10pt
%\setkomafont{disposition}{\normalfont\bfseries}	

\usepackage[utf8]{inputenc} % set input encoding (not needed with XeLaTeX)

%%% PAGE DIMENSIONS
\usepackage{geometry} % to change the page dimensions
\usepackage{amsmath}
\geometry{a4paper} % or letterpaper (US) or a5paper or....
% \geometry{margin=2in} % for example, change the margins to 2 inches all round
% \geometry{landscape} % set up the page for landscape
%   read geometry.pdf for detailed page layout information

\usepackage{booktabs}% http://ctan.org/pkg/booktabs

\usepackage{graphicx} % support the \includegraphics command and options
\usepackage{natbib} % support the \includegraphics command and options
\usepackage{xcolor,colortbl}

\usepackage{multicol}

% \usepackage[parfill]{parskip} % Activate to begin paragraphs with an empty line rather than an indent

%%% PACKAGES
\usepackage{booktabs} % for much better looking tables
\usepackage{array} % for better arrays (eg matrices) in maths
\usepackage{paralist} % very flexible & customisable lists (eg. enumerate/itemize, etc.)
\usepackage{verbatim} % adds environment for commenting out blocks of text & for better verbatim
\usepackage{subfigure} % make it possible to include more than one captioned figure/table in a single float
% These packages are all incorporated in the memoir class to one degree or another...

%%% HEADERS & FOOTERS
\usepackage{fancyhdr} % This should be set AFTER setting up the page geometry
\pagestyle{fancy} % options: empty , plain , fancy
\renewcommand{\headrulewidth}{0pt} % customise the layout...
\lhead{}\chead{}\rhead{}
\lfoot{}\cfoot{\thepage}\rfoot{}

%%% SECTION TITLE APPEARANCE
\usepackage{sectsty}
\allsectionsfont{\sffamily\mdseries\upshape} % (See the fntguide.pdf for font help)
% (This matches ConTeXt defaults)

%%% ToC (table of contents) APPEARANCE
\usepackage[nottoc,notlof,notlot]{tocbibind} % Put the bibliography in the ToC
\usepackage[titles,subfigure]{tocloft} % Alter the style of the Table of Contents
\renewcommand{\cftsecfont}{\rmfamily\mdseries\upshape}
\renewcommand{\cftsecpagefont}{\rmfamily\mdseries\upshape} % No bold!

\usepackage{titling}
\usepackage[brazil]{babel}
\usepackage{xspace}

\usepackage{framed}

\usepackage{setspace}
\onehalfspacing

\usepackage{url}

%\DeclareUnicodeCharacter{00A0}{ }

%%% END Article customizations

%%% The "real" document content comes below...

\begin{document}

%\inputencoding{latin1}\input{titulo}
% !TEX root = Projeto monografia.tex

\newcommand{\sw}{\textit{software}\xspace}
\newcommand{\iso}{ISO 29110\xspace}

\title{Projeto de dissertação}
\author{Gladistone M. Afonso}
\date{} % Activate to display a given date or no date (if empty), otherwise the current date is printed 

\maketitle

\section{Introdução}

O autor identificou na empresa DataSerra, uma pequena empresa de desenvolvimento de \sw, várias demandas de melhoria em seus processos. Anteriormente, com o objetivo de alcançar essas melhorais, o autor havia pesquisado e aplicado alguns padrões encontrados na \iso mas vários fatores levaram à paralisação e até mesmo abandono dos poucos processos que viam sendo implementados.

A \iso foi desenvolvida especificamente para pequenas empresas ou pequenas equipes de desenvolvimento de \sw, chamadas em inglês de \textit{Very Small Entities} (VSE). O autor pretende dar continuidade à implementação de seus processos utilizando o trabalho de pesquisa da dissertação como catalisador, assim como a carga de conhecimento adquirida durante o curso de Mestrado.

Como cita \cite{pressman}, a construção de \sw dentro de prazos estabelecidos e com qualidade ainda é um problema que atinge grande parte das empresas desenvolvedoras. Um dos maiores problemas envolvendo prazos e qualidade na empresa foco foi a demora ou até mesmo inexistência de \textit{feedback} aos clientes em relação à solução de problemas ou implementação de solicitações de mudanças. O autor selecionou este problema como foco do trabalho de dissertação 

\section{Objetivos gerais}

O trabalho de dissertação tem como objetivo \textbf{criar processos e ferramentas de \sw que gerenciem a comunicação com o cliente}.

\section{Objetivos específicos}

Para lograr êxito na criação da solução ideal para a demanda identificada, o autor determinou os objetivos abaixo:

\begin{itemize}

\item Realizar um diagnóstico da situação atual dos processos que envolvem comunicação com o cliente;

\item Categorizar a situação atual destes processos através da análise SWOT\footnotemark;

\footnotetext{Avaliação das forças, fraquezas, oportunidades e ameaças, dos termos em inglês \textit{strengths, weaknesses, opportunities e threats} \citep{kotler}}

\item Pesquisar a norma \iso a procura de soluções que possam ser aplicadas às ameaças e fraquezas identificadas na análise SWOT;

\item Desenvolver um projeto de melhoria de processo que contemple a criação de ferramentas de \sw que facilitem a implantação e operacionalização das melhorias;

\item Executar o projeto de melhorias;

\item Coletar os resultados do projeto de melhorias.

\end{itemize}

\section{Metodologia}

O trabalho se dará em tres etapas distintas: análise, projeto e conclusão.

A análise se dará a partir do mapeamento dos processos de produção atuais e olhar crítico em cada uma das etapas identificadas. O autor se utilizará de algumas ferramentas de administração, tais como a análise SWOT, para caracterizar e avaliar cada etapa do processo produtivo.

O projeto será a etapa onde as ações necessárias para a melhora dos processos serão elencadas, priorizadas e devidamente documentadas. Também serão identificados os principais atores (\textit{stakeholders}), o cronograma, os recursos necessários e outros elementos, conforme diretrizes do PMBOK\footnotemark{} criado pelo PMI\footnotemark. Nesta etapa serão desenvolvidas as ferramentas de \sw que buscarão facilitar as melhorias através da integração de informações vitais para os processos identificados.

\footnotetext{\textit{Project Management Book Of Knowledge}, guia de boas práticas de gerência de projetos mundialmente reconhecido}
\footnotetext{\textit{Project Management Institute}, instituição internacional referência em gerência de projetos.}

A conclusão consistirá na coleta dos resultados obtidos após a implantação do projeto de melhoria dos processos e análise da qualidade destes resultados em comparação ao cenário atual da empresa.

\section{Cronograma preliminar}

As etapas descritas anteriormente serão desempenhadas de acordo com o cronograma da Tabela \ref{tab:crono}.

\begin{table}[h!]\footnotesize
\centering
\begin{tabular}{|c|p{12cm}|}
\hline

	\textbf{Data}&
	\textbf{Marco}\\
\hline

	21/11/2014&Diagnóstico da situação do processo produtivo atual da empresa\\
\hline

	21/11/2014&Análise SWOT\\
\hline

	05/12/2014&Pesquisar a norma \iso a procura de soluções que possam ser aplicadas às ameaças e fraquezas identificadas na análise SWOT\\
\hline

	19/12/2014&Desenvolver um projeto de melhoria de processo\\
\hline

	23/02/2015&Seminário do projeto de dissertação\\
\hline

	01/03/2015&Executar o projeto de melhorias\\
\hline

	01/04/2015&Coletar os resultados do projeto de melhorias\\
\hline

	01/06/2015&Defesa da dissertação\\
\hline

\end{tabular}
\caption {Cronograma preliminar}
\label{tab:crono}
\end{table}

%\bibliographystyle{plain}
%\bibliographystyle{bibstyle/latex8}     
%\bibliographystyle{apalike-url}     
\bibliographystyle{Bibliografia/apalike-pt}     
%\bibliographystyle{lastchecked}     
            
\bibliography{Bibliografia/research}

\end{document}